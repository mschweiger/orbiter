\documentclass[Orbiter User Manual.tex]{subfiles} 
\begin{document}

\section{Demo mode}
Orbiter can be run in "demo" or "kiosk" mode to facilitate its use in public environments such as exhibitions and museums.\\
Demo mode can be configured by manually editing the Orbiter.cfg configuration file in the main Orbiter directory. The following options are available:

%\begin{table}[H]
	%\centering
	\begin{longtable}{ |p{0.25\textwidth}|p{0.05\textwidth}|p{0.6\textwidth}| }
	\hline\rule{0pt}{2ex}
	\textbf{Flag} & \textbf{Type} & \textbf{Description}\\
	\hline\rule{0pt}{2ex}
	DemoMode & bool & Set to TRUE to enable demo mode. (default: FALSE)\\
	\hline\rule{0pt}{2ex}
	BackgroundImage & bool & Set to TRUE to cover the desktop with a static image. (default: FALSE)\\
	\hline\rule{0pt}{2ex}
	BlockExit & bool & Set to TRUE to disable the Exit function in Orbiter’s Launchpad dialog. If this option is enabled, Orbiter can only be terminated via the task manager. (default: FALSE)\\
	\hline\rule{0pt}{2ex}
	MaxDemoTime & float & Maximum runtime for a simulation session [s]. Orbiter automatically returns to the Launchpad when the time has expired.\\
	\hline\rule{0pt}{2ex}
	MaxLaunchpadIdleTime & float & Maximum time spent in the Launchpad without user input before Orbiter auto-launches a demo scenario [s].\\
	\hline
	\end{longtable}
%\end{table}

\noindent
In Demo mode, only the Scenario tab is accessible in the Launchpad dialog, to prevent users from modifying simulation configuration features such as screen resolution or plug-in modules. Orbiter should be configured as required before launching into demo mode.\\
To use the auto-launch feature in demo mode, a folder Demo must be created in the main scenario folder. Orbiter randomly picks a scenario from this folder to launch.\\
\textbf{Note:} When using Orbiter in Demo mode, it is recommended that the simulation is run in a window, or in a fullscreen mode that matches the native screen resolution, to avoid excessive switching between video display modes.


\end{document}
