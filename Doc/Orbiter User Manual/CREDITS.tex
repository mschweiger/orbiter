\documentclass[Orbiter User Manual.tex]{subfiles}
\begin{document}

\section{Credits}
%TODO review content
%TODO check websites online
%TODO handle dead websites?

\subsection{Special thanks}
Doug, Josh, Gary, Orb and the entire Orbiter Forum team for keeping things running smoothly, and in particular Josh for providing and maintaining the server for the forum and Orbiter downloads.\\
Jarmo for pushing the envelope with the D3D9 client, and helping debug Orbiter and the graphics interface.\\
All beta testers and bug reporters for their help in getting the new version into shape.\\
All Orbiter users for their continued support. Keep playing!

\subsection{Source contributions and components}
\textbf{XRSound sound module}\\
Copyright (c) Doug Beachy \url{https://www.alteaaerospace.com/}\\
See Sound/XRSound/LICENSE for XRSound license.\\
\\
\textbf{TransX MFD mode}\\
Duncan Sharpe: TransX MFD mode module\\
Steve Arch: TransX development \url{http://orbiter.quorg.org}

\subsection{Libraries, code, data, algorithms}
\textbf{IrrKlang}\\
Audio and sound library\\
\url{https://www.ambiera.com/irrklang/}\\
\\
\textbf{Zlib}\\
Compression/decompression library\\
Copyright (C) 1995-2013 Jean-loup Gailly and Mark Adler\\
Jean-loup Gailly \href{mailto:jloup@gzip.org}{jloup@gzip.org}\\
Mark Adler \href{mailto:madler@alumni.caltech.edu}{madler@alumni.caltech.edu}\\
\\
\textbf{VSOP87}\\
Planetary perturbation terms for Mercury to Neptune\\
Bureau des Longitudes, CNRS URA 707\\
P. Bretagnon \href{mailto:pierre@bdl.fr}{pierre@bdl.fr}\\
G. Francou \href{mailto:francou@bdl.fr}{francou@bdl.fr}\\
\\
\textbf{Lunar Solution ELP 2000-82B}\\
Semi-analytical lunar ephemerides\\
Bureau des Longitudes, CNRS URA 707\\
75014, Paris, France\\
M. Chapront-Touze, J. Chapront\\
Astron. Astrophys. 124, 50 (1983)\\
Astron. Astrophys. 190, 342 (1988)\\
\\
\textbf{Earth precession parameters}\\
IAU SOFA C Library\\
\url{http://www.iausofa.org/}\\
\\
\textbf{Planetary precession parameters}\\
IAU/IAG Working Group\\
Report of the IAU/IAG Working Group on cartographic coordinates and rotational elements 2006, \url{http://www.springerlink.com/content/e637756732j60270/}\\
\\
\textbf{Phobos and Deimos ephemeris modules}\\
Carl Romanik ("Chode")\\
Code based on: Sinclair, Astron. Astrophys. 220, 321 (1989)\\
"Testing against Horizons shows agreement within 20km for Phobos, 50km for Deimos for 2000-2024."\\
\\
\textbf{Miranda, Ariel, Umbriel, Titania, Oberon ephemeris modules}\\
Carl Romanik ("Chode")\\
Code based on: Laskar and Jacobson, Astron. Astrophys. 188, 212 (1987)\\
"According to the Horizons documentation, this is the same theory they use for Uranus, and the agreement of the DLLs with Horizons looks to be within about 50km."\\
\\
\textbf{Triton ephemeris module}\\
Carl Romanik ("Chode")\\
Code based on: Jacobson et al., Astron. Astrophys. 247, 565 (1991)\\
"This also appears to be what Horizons use, and the DLL agrees within about 1000km."
\\
\subsubsection{Harmonic Gravity Models}
\textbf{Mercury Harmonic Gravity Model}\\
MESS160A	\textit{jgmess\_160a\_sha.tab}\\
160x160 spherical harmonic gravity model produced from Magellan radio tracking data.\\
NASA JPL\\
PDS Geosciences Node, Washington University in St. Louis\\
\url{https://pds-geosciences.wustl.edu/dataserv/gravity_models.htm}\\
\\
\textbf{Venus Harmonic Gravity Model}\\
\textit{mod\_shgj120p.a01}\\
120x120 Spherical harmonic gravity model produced from Magellan radio tracking data, modified with initial C(1,0), S(1,0), C(1,1), S(1,1), coefficients padded with zeros for used with Orbiter parser.\\
NASA JPL\\
PDS Geosciences Node, Washington University in St. Louis\\
\url{https://pds-geosciences.wustl.edu/dataserv/gravity_models.htm}\\
\\
\textbf{Earth Harmonic Gravity Model}\\
\textit{egm96\_to360.tab}\\
Modified by Ajaja for use with Orbiter parser.\\
The Development of the Joint NASA GSFC and the National Imagery and Mapping Agency (NIMA) Geopotential Model EGM96, F.G. Lemoine et al, Goddard Space Flight Center, Greenbelt, Maryland, July 1998.\\
\\
\textbf{Moon Harmonic Gravity Model}\\
165x165 Spherical harmonic gravity model produced from radio tracking of the Lunar Orbiters 1‒5, Apollo 15 and 16 subsatellites, Clementine, and the Lunar Prospector spacecraft.\\
NASA JPL\\
PDS Geosciences Node, Washington University in St. Louis\\
\url{https://pds-geosciences.wustl.edu/dataserv/gravity_models.htm}\\
\\
\textbf{Mars Harmonic Gravity Model}\\
MRO120F	\textit{jgmro\_120f\_sha.tab}\\
120x120 Spherical harmonic gravity model, produced from Mars Reconnaissance Orbiter radio tracking data.\\
NASA JPL\\
PDS Geosciences Node, Washington University in St. Louis\\
\url{https://pds-geosciences.wustl.edu/dataserv/gravity_models.htm}\\
\\
\textbf{Vesta Harmonic Gravity Model}\\
"Konopliv, A.S., Park, R.S., Asmar, S.W., and Buccino, D.R., Dawn Vesta Derived Gravity Data, NASA Planetary Data System, DAWN-A-RSS-5-VEGR-V2.0, 2017."\\
NASA JPL\\
\url{https://sbn.psi.edu/pds/resource/dawn/dwnvgravL2.html?refUrl=https\%3A\%2F\%2Fsbn.psi.edu\%2Fpds\%2Farchive\%2Fgravity.html&refName=Gravity&type=Data+Type&typeUrl=https\%3A\%2F\%2Fsbn.psi.edu\%2Fpds\%2Farchive\%2Fdata-types.html}\\
\\


\subsection{Data sources planetary textures}
\textbf{Mercury}\\
Mosaics created using MESSENGER orbital images released by NASA's Planetary Data System (PDS) (\url{http://pds-geosciences.wustl.edu/missions/messenger/index.htm}) on September 7, 2012.\\
NASA/Johns Hopkins University Applied Physics Laboratory/Carnegie Institution of Washington\\
\url{http://messenger.jhuapl.edu/the_mission/mosaics.html}\\
\\
\textbf{Mercury surface labels}\\
USGS\\
Astrogeology Research Program\\
Planetary Geomatics Group\\
Gazetteer of Planetary Nomenclature\\
\url{http://planetarynames.wr.usgs.gov/}\\
\\
\textbf{Venus surface}\\
Composite of Magellan synthetic aperture radar mosaics.\\
Jet Propulsion Laboratory Multimission Image Processing Laboratory\\
Solar System Visualization Project and Magellan science team\\
p45187.tif (5120x2560)\\
\\
\textbf{Venus clouds}\\
Björn Jónsson\\
\url{http://www.mmedia.is/~bjj}\\
\\
\textbf{Earth land surface}\\
Custom processed from Landsat 7 ETM orthorectified imagery.\\
\\
\textbf{Florida}\\
Digital Orthoimage Quarter Quads,\\
Florida Department of Environmental Protection\\
Land Boundary Information System (\url{www.labins.org})\\
Download: \url{ftp://146.201.97.137/DOQQ/2004/RGB/UTM/MrSid}\\
\\
\textbf{Earth water surface}\\
Based on NASA Visible Earth Blue Marble maps\\
\url{http://visibleearth.nasa.gov/}\\
\\
\textbf{Earth night lights}\\
Custom processed, based on NASA Visible Earth Blue Marble maps\\
\url{http://visibleearth.nasa.gov/}\\
\\
\textbf{Earth clouds}\\
NASA Visible Earth Blue Marble maps\\
\url{http://visibleearth.nasa.gov/}\\
\\
\textbf{Earth elevation}\\
SRTM 90m Digital Elevation Data by NASA, released by USGS\\
CGIAR-CSI version 4 Processed for void removal by International Centre for Tropical Agriculture (CIAT)\\
\\
\textbf{Moon surface}\\
LRO LROC-WAC Global Mosaic 100m June2013\\
Arizona State University\\
Astrogeology Science Center\\
\\
\textbf{Moon elevation}\\
LOLA-GDR/Cylindrical\\
\url{http://imbrium.mit.edu/DATA/LOLA_GDR/CYLINDRICAL/IMG/}\\
\\
\textbf{Mars surface}\\
Custom processed, based on NASA MGS/MOC. 256 ppd/230m Mars Odyssey THEMIS-IR Day Global Mosaic 100m v12\\
\url{http://astrogeology.usgs.gov/search/map/Mars/Odyssey/THEMIS-IR-Mosaic-ASU/Mars_MO_THEMIS-IR-Day_mosaic_global_100m_v12}\\
Viking MDIM2.1 Colorized Global Mosaic 232m\\
\url{http://astrogeology.usgs.gov/search/details/Mars/Viking/MDIM21/Mars_Viking_MDIM21_ClrMosaic_global_232m/cub}\\
\\
\textbf{Mars elevation}\\
MOLA Mars elevation data at 128 pixels per degree\\
\url{http://pds-geosciences.wustl.edu/mgs/mgs-m-mola-5-megdr-l3-v1/mgsl_300x/meg128/}\\
\\
\textbf{Mars surface labels}\\
USGS\\
Astrogeology Research Program\\
Planetary Geomatics Group\\
Gazetteer of Planetary Nomenclature\\
\url{http://planetarynames.wr.usgs.gov/}\\
\\
\textbf{Vesta surface}\\
Dawn FC HAMO Global Mosaic 60mp\\
\url{http://astrogeology.usgs.gov/search/details/Vesta/Dawn/DLR/HAMO/Vesta_Dawn_FC_HAMO_Mosaic_Global_74ppd/cub}\\
\\
\textbf{Vesta elevation}\\
Dawn HAMO DTM Global 93mp\\
\url{http://astrogeology.usgs.gov/search/details/Vesta/Dawn/DLR/HAMO/Vesta_Dawn_HAMO_DTM_DLR_Global_48ppd/cub}\\
\\
\textbf{Jupiter}\\
"Cassini's best map of Jupiter"\\
NASA/JPL/Space Science Institute\\
Cassini Imaging Central Laboratory for Operations\\
\url{http://www.ciclops.org/view/1270/Cassinis-Best-Maps-of-Jupiter}\\
Rolf Keibel: Jupiter cloud map based on CICLOPS maps\\
\\
\textbf{Io Surface}\\
Based on:\\
Io Galileo SSI/Voyager Color Merged Global Mosaic 1km\\
Astrogeology Science Center\\
USGS\\
\url{http://astrogeology.usgs.gov/search/map/Io/Voyager-Galileo/Io_GalileoSSI-Voyager_Global_Mosaic_ClrMerge_1km}\\
\\
\textbf{Io Elevation}\\
Based on:\\
Oliver L. White, Paul M. Schenk, Francis Nimmo, Trudi Hoogenboom, "A new stereo topographic map of Io: Implications for geology from global to local scales", Journal of Geophysical Research: Planets 119(6), 1276-1301 (2014), doi: 10.1002/2013JE004591\\
\url{http://onlinelibrary.wiley.com/doi/10.1002/2013JE004591/abstract}\\
\\
\textbf{Europa}\\
Based on:\\
Europa Voyager and Galileo SSI Global Mosaic 500m\\
Astrogeology Science Center\\
USGS\\
\url{http://astrogeology.usgs.gov/search/map/Europa/Voyager-Galileo/Europa_Voyager_GalileoSSI_global_mosaic_500m}\\
\\
\textbf{Ganymede}\\
Ganymede Voyager and Galileo Color Global Mosaic 1.4km\\
Astrogeology Science Center\\
USGS\\
\url{http://astrogeology.usgs.gov/search/map/Ganymede/Voyager-Galileo/Ganymede_Voyager_GalileoSSI_Global_ClrMosaic_1435m}\\
\\
\textbf{Callisto}\\
Based on:\\
Callisto Galileo/Voyager Global Mosaic 1km\\
Astrogeology Science Center\\
USGS\\
\url{http://astrogeology.usgs.gov/search/map/Callisto/Voyager-Galileo/Callisto_Voyager_GalileoSSI_global_mosaic_1km}\\
\\
\textbf{Io, Europa, Ganymede, Callisto surface labels}\\
USGS\\
Astrogeology Research Program\\
Planetary Geomatics Group\\
Gazetteer of Planetary Nomenclature\\
\url{http://planetarynames.wr.usgs.gov/}\\
\\
\textbf{Saturn}\\
Björn Jónsson\\
\url{http://www.mmedia.is/~bjj}\\
Rolf Keibel: texture adaptation\\
\\
\textbf{Saturn rings}\\
Björn Jónsson\\
\url{http://www.mmedia.is/~bjj}\\
\\
\textbf{Mimas}\\
NASA/JPL-Caltech/SSI/Lunar and Planetary Institute\\
Cassini Imaging Central Laboratory for Operations\\
PIA 18437\\
\url{http://www.ciclops.org/view/7963/Color-Maps-of-Mimas---November-2014}\\
\\
\textbf{Enceladus}\\
NASA/JPL-Caltech/SSI/Lunar and Planetary Institute\\
Cassini Imaging Central Laboratory for Operations\\
PIA 18435\\
\url{http://www.ciclops.org/view/7961/Color-Maps-of-Enceladus---November-2014}\\
\\
\textbf{Tethys}\\
NASA/JPL-Caltech/SSI/Lunar and Planetary Institute\\
Cassini Imaging Central Laboratory for Operations\\
PIA 18439\\
\url{http://www.ciclops.org/view/7965/Color-Maps-of-Tethys---November-2014}\\
\\
\textbf{Dione}\\
NASA/JPL-Caltech/SSI/Lunar and Planetary Institute\\
Cassini Imaging Central Laboratory for Operations\\
PIA 18434\\
\url{http://www.ciclops.org/view/7960/Color-maps-of-Dione---November-2014}\\
\\
\textbf{Rhea}\\
NASA/JPL-Caltech/Space Science Institute/Lunar and Planetary Institute\\
\url{http://photojournal.jpl.nasa.gov/catalog/PIA18438}\\
\\
\textbf{Titan surface}\\
NASA/JPL/Space Science Institute/Cassini Data Analysis Program/USGS Astrogeology Science Center/Ian Regan\\
Updated, amended and restored version of USGS photomosaic\\
\url{https://astrogeology.usgs.gov/search/map/Titan/Cassini/Global-Mosaic/Titan_ISS_P19658_Mosaic_Global_4km}\\
\url{https://astrogeology.usgs.gov/search/map/Titan/Cassini/Global-Mosaic/Titan_ISS_Globe_65Sto45N_450M_AvgMos}\\
\url{https://www.flickr.com/photos/10795027@N08/43023455582/}\\
\url{https://www.insaturnsrings.com/titan-seam-blending}\\
Map used with permission\\
\\
\textbf{Titan elevation}\\
P. Corlies, A. G. Hayes, S. P. D. Birch, R. Lorenz, B. W. Stiles, R. Kirk, V. Poggiali, H. Zebker, L. Iess "Titan's Topography and Shape at the End of the Cassini Mission", Geophysical Research Letters 44(23), 11754-11761 (2017)\\
\url{https://agupubs.onlinelibrary.wiley.com/doi/full/10.1002/2017GL075518}\\
\\
\textbf{Titan surface labels}\\
USGS\\
Astrogeology Research Program\\
Planetary Geomatics Group\\
Gazetteer of Planetary Nomenclature\\
\url{http://planetarynames.wr.usgs.gov/}\\
\\
\textbf{Iapetus}\\
NASA/JPL-Caltech/SSI/Lunar and Planetary Institute\\
Cassini Imaging Central Laboratory for Operations\\
PIA 18436\\
\url{http://www.ciclops.org/view/7962/Color-Maps-of-Iapetus---November-2014}\\
\\
\textbf{Phoebe}\\
NASA/JPL/Space Science Institute\\
Cassini Imaging Central Laboratory for Operations\\
PIA 07775\\
\url{http://www.ciclops.org/view/1743/Map-of-Phoebe---December-2005}\\
\\
\textbf{Uranus}\\
James Hastings-Trew\\
\url{http://apollo.spaceports.com/~jhasting/}\\
Rolf Keibel: texture adaptation\\
\\
\textbf{Miranda, Ariel, Umbriel, Titania, Oberon}\\
Robert Stettner\\
Credits: Planetary Satellite Mean Orbital Parameters and Moon Maps\\
\\
\textbf{Neptune}\\
James Hastings-Trew\\
\url{http://apollo.spaceports.com/~jhasting/}\\
\\
\textbf{Triton, Proteus, Nereid}\\
Robert Stettner\\
Credits: Planetary Satellite Mean Orbital Parameters and Moon Maps\\
Rolf Keibel: Triton texture adaptation from Voyager images


\subsection{Celestial sphere}
\textbf{Background star databases}\\
ESA, ed. 1997, The Hipparcos and Tycho Catalogues, SP No. 1200 (ESA)\\
\url{https://www.cosmos.esa.int/web/hipparcos/hipparcos-2}\\
\url{http://cdsarc.u-strasbg.fr/ftp/cats/I/239/}\\
F. van Leeuwen (2005). A new reduction of the raw Hipparcos data, Astron. Astrophys. 439(2): 791-803. 10.1051/0004-6361:20053193\\
\url{https://www.researchgate.net/publication/1760034_Validation_of_the_new_Hipparcos_reduction}\\
\\
\textbf{Background map: Starmap 2020}\\
NASA/Goddard Space Flight Center Scientific Visualization Studio.\\
Gaia DR2: ESA/Gaia/DPAC.\\
\url{https://svs.gsfc.nasa.gov/4851}\\
\\
\textbf{Background maps: DDS2 (visible), Hydrogen alpha, IRAS (far IR), Planck (Microwave, Source: ESA/Planck), Radio, RASS (X-ray), Fermi (Gamma)}\\
Chromoscope \url{http://www.chromoscope.net/}\\
Stuart Lowe, Chris North (Cardiff University) and Robert Simpson (Oxford University)\\
\\
%TODO name doesn't match what is shown in launchpad
\textbf{Background map: WMAP Microwave images}\\
WMAP Science Team\\
WMAP "Science on a sphere" microwave sky images\\
NASA/LAMBDA\\
\url{http://lambda.gsfc.nasa.gov/product/map/current/sos/}\\
\\
\textbf{Constellation boundaries}\\
(c) 2012-2022 Pierre Barbier,\\
Interpolated merged edges (J2000)\\
Creative Commons Attribution-ShareAlike 4.0 International License\\
\url{https://pbarbier.com/constellations/boundaries.html}\\
\url{https://pbarbier.com/constellations/lines_in_20.txt}\\


\subsection{Spacecraft and structure models and textures}
\textbf{DeltaGlider and DG-S mesh and virtual cockpit}\\
Roger "Frying Tiger" Long\\
\\
\textbf{Space Shuttle Atlantis}\\
Michael Grosberg: meshes and textures\\
Don Gallagher: mesh and texture extensions\\
Robert Conley ("estar"): Module extensions: Movable arm and grappling, including MMU and Satellite extensions; documentation\\
David Hopkins: Module code extensions\\
Damir Gulesich: Space Shuttle External Tank and Solid Rocket Booster mesh and textures.\\
\\
\textbf{LDEF mesh and textures}\\
Don Gallagher\\
\\
\textbf{ISS model "Project Alpha"}\\
Andrew Farnaby\\
\\
\textbf{Mir model}\\
Jason Benson ("agent036")\\
\\
\textbf{Dragonfly model}\\
Roger "Frying Tiger" Long: Mesh improvements and textures\\
Radu Poenaru: Electrical and environmental simulation, Dragonfly panels\\
\\
\textbf{Shuttle-A model}\\
Roger "Frying Tiger" Long: Shuttle-A mesh\\
Radu Poenaru: Virtual cockpit and cargo management\\
\\
\textbf{Hubble Space Telescope (HST) model}\\
David Sundstrom\\
\\
\textbf{KSC VAB mesh}\\
Valerio Oss\\
\\
\textbf{PTV (Personal transport vehicle) mesh}\\
Balázs Patyi\\
\href{mailto:patyibalazs@yahoo.com}{patyibalazs@yahoo.com}\\
\\
\textbf{Default exhaust texture, cloud microtextures}\\
"McWgogs"\\
\url{http://mcwgogs.deviantart.com/}


\end{document}
