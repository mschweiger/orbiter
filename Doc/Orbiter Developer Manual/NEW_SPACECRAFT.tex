\documentclass[Orbiter Developer Manual.tex]{subfiles}
\begin{document}

\section{How to create a new spacecraft class}
To add a new spacecraft class to Orbiter, the following steps must be performed:

\begin{itemize}
\item Define the physical parameters of the new spacecraft class in a configuration file in the Config subdirectory.
\item Create a surface mesh which defines the ship's visual appearance, in the Meshes subdirectory.
\item Optionally, add any textures used by the vessel, to the Textures subdirectory.
\item Add a scenario which includes one or more spacecraft of the new class in the Scenarios subdirectory.
\end{itemize}

\noindent
The above steps allow you to create a basic ship with generic parameters. To fully customize your new spacecraft one more step is required:

\begin{itemize}
\item Add a DLL module for the new vessel class which customises its behaviour (moving parts, custom cockpit panels, custom flight model, etc.) in the Modules subdirectory.
\end{itemize}

Refer to the OrbiterAPI document for information on writing vessel modules for Orbiter. A number of sample modules with source code is contained in the Orbitersdk\textbackslash samples folder.\\
\\
Note that every spacecraft class \textit{must} have a configuration file, even if all its parameters are defined in a DLL module. (In this case, the only entry in the configuration file may be the module name.)

\end{document}
