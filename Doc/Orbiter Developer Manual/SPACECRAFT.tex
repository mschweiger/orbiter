\documentclass[Orbiter Developer Manual.tex]{subfiles}
\begin{document}

\section{Spacecraft design}
This section describes how to create a new \textit{vessel class} for Orbiter by writing a \textit{vessel DLL module}. Although it is possible to create simple vessel classes by writing a vessel configuration file without a custom module, the full potential of Orbiter's custom spacecraft design capabilities can only be realised with a specialised module.\\

\alertbox{All vessels of a given class share the same DLL module. Orbiter only loads a single instance of the DLL. This means that global variables are shared between all vessels of that class. Do not store data which are specific to individual vessels in global or static variables, because they can be overwritten by another vessel.}


\subsection{Module initialisation}
When the user launches the simulation by picking a scenario from the Orbiter Launchpad dialog and pressing the "Launch Orbiter" button, Orbiter will load the vessel DLL module for each spacecraft type used in the simulation, and call its \textit{InitModule} function. This function is called only once per Orbiter session, no matter how many spacecraft of that type appear in the simulation. It will not be called again if the user exits the simulation to the Launchpad, and reloads another simulation scenario. You can use it to initialise global (non-instance specific and non-session specific) parameters.

\begin{lstlisting}
#define ORBITER_MODULE
#include "orbitersdk.h"

HINSTANCE g_hDLL;

DLLCLBK void InitModule( HINSTANCE hModule )
{
	g_hDLL = hModule;
	// perform global module initialisation here
}
\end{lstlisting}

\noindent
In this example, we use the \textit{InitModule} function to save the module instance handle passed to the function in global variable \textit{g\_hDLL}. This handle is useful later, e.g. when loading resources stored in the module file. Note the first line of the code example, which defines the \textit{ORBITER\_MODULE} flag. This flag should be included in all Orbiter DLL modules, to ensure proper execution of initialisation and cleanup functions.\\
\\
At the end of a simulation run, Orbiter calls the \textit{ExitModule} function for each DLL module.

\begin{lstlisting}
DLLCLBK void ExitModule( HINSTANCE hModule )
{
	// perform module cleanup here
}
\end{lstlisting}

\noindent
If you performed any dynamic memory allocation in \textit{InitModule}, this is a good place to perform the corresponding cleanup operations which de-allocate that memory.


\subsection{Vessel initialisation}
\label{ssec:vessel_init}
To allow initialisation of individual spacecraft, Orbiter will call the ovcInit function each time a scenario is loaded, for each vessel of that type listed in the scenario file. Orbiter will also call \textit{ovcInit} during the simulation if a new vessel of this type is created. The main purpose of ovcInit is to create an instance of a \textit{VESSEL}-derived interface class. \textit{VESSEL} is a class defined in the Orbiter API which is the primary means of communication between Orbiter and your own spacecraft class. In order to make use of the interface, you should derive your own vessel class derived from \textit{VESSEL}. In \textit{ovcInit}, you then create an instance of that class and return it back to Orbiter. Note that in the latest Orbiter version, the new \textit{VESSEL2} class has been introduced which inherits all the methods of \textit{VESSEL}, and introduces a number of new callback functions which replace the previous method of event notification. You should derive your vessel class from \textit{VESSEL2} to make use of this latest interface.\\
\\
As an example, let's create a new class called \textit{MyVessel}, and create an instance in \textit{ovcInit}:

\begin{lstlisting}
class MyVessel: public VESSEL2
{
	public:
		MyVessel( OBJHANDLE hObj, int fmodel ) : VESSEL2( hObj, fmodel ) {}
		~MyVessel() {}
		// add more vessel methods here
};

DLLCLBK VESSEL* ovcInit( OBJHANDLE hvessel, int flightmodel )
{
	return new MyVessel( hvessel, flightmodel );
}
\end{lstlisting}

\noindent
\textit{ovcInit} passes two parameters to your module: a handle to the vessel for which you are about to create an interface, and a flag for the type of flight model requested by the user. Both parameters are passed on to the vessel constructor. The vessel handle is required to identify your vessel when requesting information from Orbiter. The \textit{flightmodel} flag can be used to implement different behaviour in your module, for example to define an "easy" and a "complex" flight model, which can then be selected by the user. You don't need to store these parameters in your module, because you can retrieve them with the \textit{GetHandle} and \textit{GetFlightModel} methods of the \textit{VESSEL} class.\\
\\
To ensure proper cleanup at the end of a simulation session, you must implement the \textit{ovcExit} function to delete your vessel:

\begin{lstlisting}
DLLCLBK void ovcExit( VESSEL* vessel )
{
	if (vessel) delete (MyVessel*)vessel;
}
\end{lstlisting}

\noindent
Note that you need to cast the generic \textit{VESSEL} pointer passed by Orbiter to your own vessel class to ensure that the correct destructors are called.


\subsection{Reading and saving a vessel state}
Next, you need to make sure that your vessel is able to read its initial state from a scenario file at the start of a simulation, and to save its state in a scenario at the end of the simulation. This is done by overloading the \textit{clbkLoadStateEx} and \textit{clbkSaveState} methods of the \textit{VESSEL2} class. Note that you only need to overload these methods if your vessel requires nonstandard parameters to be stored in the scenario file. Standard parameters (such as position or velocity) are automatically read and written by the base class methods.

\begin{lstlisting}
class MyVessel: public VESSEL2
{
	public:
		MyVessel( OBJHANDLE hObj, int fmodel ): VESSEL2( hObj, fmodel ) {}
		~MyVessel() {}
		void clbkLoadStateEx( FILEHANDLE scn, void* status );
		void clbkSaveState( FILEHANDLE scn );
	private:
		double myparam;
};

void MyVessel::clbkLoadStateEx( FILEHANDLE scn, void* status )
{
	char *line;

	while (oapiReadScenario_nextline( scn, line ))
	{
		if (!strnicmp( line, "MYPARAM", 7 ))
		{
			sscanf( line + 7, "%lf", &myparam );
		}
		else
		{
			ParseScenarioLineEx( line, status );
		}
	}
}

void MyVessel::clbkSaveState( FILEHANDLE scn )
{
	VESSEL2::clbkSaveState( scn );
	oapiWriteScenario_float( scn, "MYPARAM", myparam );
}
\end{lstlisting}

\noindent
In the code fragment above, we use the overloaded \textit{clbkLoadStateEx} function to read myparam from the scenario, were it is stored under the \textit{MYPARAM} label. The function reads each line of the scenario file associated with our vessel, using the \textit{oapiReadScenario\_nextline} function. In the loop, we process the \textit{MYPARAM} line, and pass everything else to Orbiter via \textit{ParseScenarioLineEx} for default processing. Likewise, in \textit{clbkSaveState}, the base class method \textit{VESSEL2::clbkSaveState} is called to store all default parameters, before writing our private \textit{MYPARAM} value. Of course, a real vessel implementation may need to store a large number of parameters in the scenario to make sure its status is completely defined when the scenario is loaded next time.


\subsection{Defining class capabilities}
One of the most important callback functions that should be overloaded is the \textit{clbkSetClassCaps} method. It defines the general capabilities and properties of your spacecraft, e.g. its mass, size, visual representation, engine layout etc.

\begin{lstlisting}
void MyVessel::clbkSetClassCaps( FILEHANDLE cfg )
{
	SetEmptyMass( 1000.0 );
	SetSize( 10.0 );
	AddMesh( oapiLoadMeshGlobal( "MyVessel.msh" ) );
	// define vessel capabilities here
}
\end{lstlisting}

\noindent
In the above example, we define a few essential parameters (empty mass and mean radius), and load a mesh to provide a visual representation for our new spacecraft class. In practical applications, many more parameters may have to be defined here. Note that the file handle passed to the function points to the configuration file (.cfg) of the vessel. This can be used to read parameters from the file, thereby allowing the user to overwrite parameters by editing the configuration file.\\
\\
We now have a "skeleton implementation" for our new spacecraft class. To make it interesting, many more properties need to be defined, such as rocket engines (or air-breathing engines), aerodynamic properties, animations, etc. Some of these aspects are described in the rest of this chapter. For a complete (and sometimes quite complex) vessel implementations, see the sample projects in the \textit{Orbitersdk\textbackslash samples} subdirectory.



\subsection{Creating rocket engines}
To propel your ship in space, you must equip it with engines. There exist a variety of different rocket engine types, such as liquid and solid fuel engines, or more futuristic ones such as ion or photon drives.

\subsubsection{A bit of theory}
\textbf{Thrust force}\\
\\
Despite their very different design, all engines work by the same principle: generating a thrust force in one direction by expelling particles in the opposite direction at high velocity. A liquid-fuel engine, for example, consists of a burn chamber in which a mixture of propellant and oxydiser are ignited, and a nozzle through which the expanding gas is forced at high velocity. The force $F_{th}$ generated by the engine is proportional to the propellant mass flow dm/dt and the velocity $v_{0}$ of the expelled gas:

\[ \vec{F}_{th} = \frac{dm}{dt} (t) \vec{v}_{0}\]

\noindent
When creating a thruster, you need to specify the maximum force $F_{th}$ it can generate when it is driven at full power, and the propellant exit velocity $v_{0}$. (in Orbiter, $v_{0}$ is called the \textit{fuel-specific impulse}, or Isp). The Isp value determines how much fuel per second is consumed to obtain a given thrust force. The higher the Isp value, the more fuel-efficient the engine.\\

\begin{figure}[H]
  \centering
  \includegraphics[width=0.5\hsize]{engine.png}
\end{figure}

\noindent
Sometimes the \textit{thrust-specific fuel consumption} (TSFC) is quoted in the literature. This is the amount of propellant that needs to be burned per second to obtain 1N of thrust. Thus the TSFC is the inverse of the Isp and has units of [$s \ m^{-1}$], or more intuitively [$kg \ s^{-1} N^{-1}$].\\
\\
\textbf{Note:} In Orbiter, the thrust is specified as a force, and has units of Newton [$1N = 1kg \ m \ s^{-2}$]. In the literature, thrust is often specified in units of kg. To convert such data into Orbiter units, multiply by $1g = 9.81 \ m \ s^{-2}$. Isp is specified as a velocity in Orbiter, with units of $m \ s^{-1}$. In the literature it is often given in units of seconds [s]. To convert to Orbiter units, again multiply by 1g.\\
\\
\textbf{How long will my fuel last?}\\
\\
The burn time $T_{b}$ at full thrust $F_{max}$ for fuel mass $m_{F}$ is given by

\[ T_{b} = \frac{m_{F} \ Isp}{F_{max}} \]

\noindent
\textbf{Pressure-dependent thrust efficiency}\\
\\
Most conventional rocket engines work less efficiently in the presence of ambient atmospheric pressure, because the ignited gas must be expelled through the nozzle against the outside pressure of the atmosphere. This leads to a reduction of the thrust force at ambient pressure \textit{p}:

\[ F(p) = F_{0} - p A \]

\noindent
where $F_{0}$ is the vacuum thrust rating and A has units of an area [$m^{2}$] and can be regarded as the \textit{effective nozzle cross section}. If we know the force $F_{1}$ generated at ambient pressure $p_{1}$, then

\[ F_{1} = F_{0} - p_{1}A \ \Rightarrow \ A = \frac{F_{0} - F_{1}}{p_{1}} \]

\noindent
and therefore

\[ F(p) = F_{0} - p\frac{F_{0} - F_{1}}{p_{1}} = F_{0}(1 - \frac{F_{0} - F_{1}}{F_{0}p_{1}}) = F_{0}(1 - p\eta) \]

\noindent
and likewise

\[ Isp(p) = Isp_{0}(1 - p\eta) \]

\noindent
In the literature, the pressure-dependency of engine thrust is often defined by specifying the Isp value for both vacuum and a given reference pressure (e.g. atmospheric pressure at sea level). Orbiter uses the same convention to apply pressure dependency.\\
\\
\textbf{Thrust level}\\
In Orbiter, thrusters can be driven at any level L between 0 (cutout) and 1 (full thrust). The actual thrust force generated by the engine is thus calculated as

\[ F(p) = F_{max}(p) \cdot L \]

\noindent
In reality, thrusters can often only be driven at maximum, or within a limited range below maximum. This is not currently implemented in Orbiter, but may be introduced in a future version.\\
\\
\textbf{Thruster placement and thrust direction}\\
\\
The effect of a thruster depends on its placement on the vessel, and the direction in which the thrust force is generated. In the most general case, a thruster will produce both a linear acceleration (due to a force) and an angular acceleration (do to torque).\\
Torque is generated if the force vector does not pass through the vessel's centre of gravity (CG)

\begin{figure}[H]
  \centering
  \includegraphics[width=0.5\hsize]{rocket_thrust.png}
\end{figure}

\noindent
The torque is then given by the cross product

\[ \vec{M} = \vec{F} \times \vec{r} \]

\noindent
(remember that Orbiter uses a left-handed coordinate system!) To avoid uncontrollable spin you should design your ship's main engines so that their force vector passes through the CG. Vessel coordinates are always defined so that the CG is at the origin (0,0,0). Therefore, a thruster located at (0,0,-10) and generating thrust in direction (0,0,1) would not generate torque.\\
\\
\textbf{Attitude thrusters: Rotation}\\
\\
Sometimes generating torque is desired in order to rotate the spacecraft. For controlled attitude manouevres one then usually wants to change only the angular moment, without also inducing a linear acceleration. This requires the simultaneous operation of at least 2 thrusters so that their linear moments cancel.

\begin{figure}[H]
  \centering
  \includegraphics[width=0.5\hsize]{rocket_rotation.png}
\end{figure}

\noindent
\textbf{Attitude thrusters: Translation}\\
\\
In order to provide small linear accelerations in various directions (for example, to line the ship up with the docking port of a space station), thrusters must be driven single or in groups so that they don't generate torque. Sometimes it is possible to re-use the rotational attitude thrusters for this task, but it is equally possible to add separate linear thrusters.\\

\begin{figure}[H]
  \centering
  \includegraphics[width=0.5\hsize]{rocket_translation.png}
\end{figure}

\noindent
\textbf{Engine gimbal and thrust vectoring}\\
\\
Using attitude thrusters in a launch vehicle during the burn phase of the main engines is usually not practical. Instead, attitude control is performed by tilting the main engines and thereby generating a torque as described above. In practice this may be done by suspending the engines in a gimbal system which allows rotation around one or two axes. In Orbiter, this can be implemented by modifying the thrust direction of the engine.\\
\\
Another way to change the thrust direction is by inserting deflector plates into the exhaust stream.\\
\\
\textbf{Torque, angular momentum and angular velocity}\\
\\
The relationship between torque M and angular velocity is given by Euler's equations for a rotating rigid body:
\[ J_{x}\dot{\omega}_{x} = M_{x} - (J_{z} - J_{y})\omega_{y}\omega_{z} \]
\[ J_{y}\dot{\omega}_{y} = M_{y} - (J_{x} - J_{z})\omega_{z}\omega_{x} \]
\[ J_{z}\dot{\omega}_{z} = M_{z} - (J_{y} - J_{x})\omega_{x}\omega_{y} \]

\noindent
where ($J_{x}$, $J_{y}$, $J_{z}$) are the principal moments of the inertia tensor (PMI), ($M_{x}$, $M_{y}$, $M_{z}$) are the components of the torque tensor, and ($\omega_{x}$, $\omega_{y}$, $\omega_{z}$) are the angular velocity components around the x, y, and z-axes. In Orbiter, this system of differential equations is solved by a trapezoid rule.


\subsubsection{Putting it all into the module}
Now that you know how thrusters work, it is time to add a few to your new ship. As with other vessel capabilities, thrusters should usually be designed in the \textit{clbkSetClassCaps} callback function, for example like this (assuming that \textit{MyVessel} is a class derived from \textit{VESSEL2}):

\begin{lstlisting}
void MyVessel::clbkSetClassCaps( FILEHANDLE cfg )
{
	// vessel caps definitions
}
\end{lstlisting}

\noindent
\textbf{Propellant resources}\\
\\
Thrusters can only be operated if they are connected to propellant resources (e.g. fuel tanks). To create a propellant resource:

\begin{lstlisting}
class MyVessel: public VESSEL
{
	...
	PROPELLANT_HANDLE ph_main;
}

void MyVessel::clbkSetClassCaps( FILEHANDLE cfg )
{
	...
	const double MAX_MAIN_FUEL = 1e5;
	ph_main = CreatePropellantResource( MAX_MAIN_FUEL );
	...
}
\end{lstlisting}

\noindent
which creates a fuel tank of capacity 10$^{5}$kg. \textit{CreatePropellantResource} returns a handle to the new tank, which is used later to connect thrusters to the tank.\\
\\
\textit{CreatePropellantResource} accepts two further optional parameters: the initial fuel mass, and a fuel efficiency factor \textit{eff} between 0 and 1. By default, the tank is full, with fuel efficiency 1. If an \textit{eff} < 1 is specified, then the thrust force generated by all connected thrusters is modified by

\[ F' = F \cdot eff \]

\noindent
\textbf{Creating thrusters}\\
\\
To add a new thruster, use the \textit{CreateThruster} command:
\begin{lstlisting}
class MyVessel: public VESSEL
{
	...
	THRUSTER_HANDLE th_main;
}

void MyVessel::clbkSetClassCaps( FILEHANDLE cfg )
{
	...
	const double MAX_MAIN_THRUST = 2e5;
	const double VAC_MAIN_ISP = 4200.0;
	th_main = CreateThruster( _V( 0, 0, -8 ), _V( 0, 0, 1 ),
		MAX_MAIN_THRUST, ph_main, VAC_MAIN_ISP );
	...
}
\end{lstlisting}

\noindent
This adds a thruster at position (0,0,-8) with a thrust vector in the positive z-direction, with the specified max. thrust and Isp values, and connected to the tank we added earlier. In this configuration, the engine efficiency is assumed not to be affected by atmospheric pressure. For increased realism, we could introduce pressure-dependency by adding an additional Isp value at a reference pressure, and the reference pressure itself:


\begin{lstlisting}
void MyVessel::clbkSetClassCaps( FILEHANDLE cfg )
{
	...
	const double MAX_MAIN_THRUST = 2e5;
	const double VAC_MAIN_ISP = 4200.0;
	const double NML_MAIN_ISP = 3500.0;
	const double P_NML = 101.4e3;
	th_main = CreateThruster( _V( 0, 0, -8 ), _V( 0, 0, 1 ),
		MAX_MAIN_THRUST, ph_main, VAC_MAIN_ISP, NML_MAIN_ISP, P_NML );
	...
}
\end{lstlisting}

\noindent
This reduces the Isp value at sea level to 3500 and performs a linear interpolation to obtain the Isp at arbitrary pressures. Note that we could have omitted the last parameter, \textit{P\_NML}, because the reference pressure defaults to 101.4 kPa (atmospheric pressure at Earth sea level).\\
\\
If you descend into a very dense planetary atmosphere, Orbiter will extrapolate the Isp value beyond sea level pressure, until Isp drops to zero. At this point, the thruster will stop working altogether.\\
\\
\textbf{Grouping thrusters}\\
\\
Although it is possible to address thrusters individually in your module, it is often easier to engage them in groups. Groups are also required to activate manual user thruster control via the keyboard or joystick, and the automatic navigation modes such as \textit{killrot}.\\
\\
% TODO handle API doc reference
Orbiter has a number of standard thruster groups, such as \textit{THGROUP\_MAIN}, \textit{THGROUP\_RETRO}, \textit{THGROUP\_HOVER}, and a full set of attitude thruster groups. For a full listing, see \textit{VESSEL::\-Create\-Thruster\-Group} in the API Reference Manual.\\
\\
It is the responsibility of the vessel designer to make sure that thrusters are grouped in a sensible way. For example, whenever the user presses \keystroke{+}$_{Num}$, all thrusters in \textit{THGROUP\_MAIN} will fire. If the thrusters grouped in \textit{THGROUP\_MAIN} behave in an unexpected or non-intuitive way it will be confusing to the user. Furthermore, if attitude thrusters are not appropriately grouped, some or all of the navigation modes may fail.\\
\\
To group thrusters, use the \textit{CreateThrusterGroup} command:

\begin{lstlisting}
void MyVessel::clbkSetClassCaps( FILEHANDLE cfg )
{
	...
	thg_main = CreateThrusterGroup( th_main, 2, THGROUP_MAIN );
	...
}
\end{lstlisting}

\noindent
(this assumes that \textit{th\_main} is an array of 2 thruster handles which have been created previously). The function returns a handle to the group which can be used later to address the group.\\
\\
Apart from the standard groups, Orbiter allows to create custom groups by using the \textit{THGROUP\_USER} label. Custom groups are not engaged by any of the standard manual or automatic control methods, therefore the module must implement a suitable control interface for these groups.


\subsubsection{Defining exhaust flames}
When you define a thruster with \textit{CreateThruster}, Orbiter will not automatically generate visuals for the exhaust flames when the thruster is engaged. Sometimes exhaust flames may not be appropriate, or, more importantly, you may want to detach the \textit{logical} thruster definition from the physical definition (more about this below).\\
\\
To create an exhaust flame definition use the \textit{AddExhaust} function. \textit{AddExhaust} comes in two flavours:
\begin{itemize}
\item \textit{UINT AddExhaust( THRUSTER\_HANDLE th, double lscale, double wscale, SURFHANDLE tex = 0 ) const}
\item \textit{UINT AddExhaust( THRUSTER\_HANDLE th, double lscale, double wscale, const VECTOR3 \&pos, const VECTOR3 \&dir, SURFHANDLE tex = 0 ) const}
\end{itemize}

\noindent
Both versions require a handle to the logical thruster they are linked to, and two size parameters (longitudinal and transversal scaling), but while the first version takes exhaust location and direction directly from the thruster definition, the second version gets location and direction passed as parameters.\\
\\
Here is an example demonstrating how you would use the second version of \textit{AddExhaust}:\\
\\
Let's assume you build a rocket propelled by 4 main engines arranged in a regular square pattern. The engines have fixed orientation (no individual gimbal mode) and all thrust force vectors are parallel. In addition, the engines produce identical thrust magnitudes at all times.\\
\\
Then the 4 engines can be represented by a single logical thruster, whose magnitude is the sum of the 4 actual engines, and positioned in the geometric centre. This simplifies the code, and is more efficient, because Orbiter does not need to add up 4 individual force vectors.\\
\\
However, you still want to see exhaust flames for each of the 4 engines, so you would use the second version of \textit{AddExhaust} to define 4 exhaust flames at the correct positions.\\
\\
The disadvantage of the second version is that changes in the position or orientation of the thruster (for example as a result of \textit{SetThrusterPos} or \textit{SetThrusterDir}) are not automatically propagated to the exhaust flames. Therefore, if you plan to move or tilt the thrusters, you should create them individually and use the first version of \textit{AddExhaust}.\\
\\
\textbf{Custom exhaust textures}\\
\\
By default, Orbiter uses a standard texture to render exhaust flames. If you want to customise the exhaust appearance on a per-thruster basis, you can pass a nonzero surface handle tex to both of the \textit{AddExhaust} versions. To obtain a surface handle for a custom texture, use the \textit{oapiRegisterExhaustTexture} function.

\begin{lstlisting}
...
SURFHANDLE tex = oapiRegisterExhaustTexture( "MyExhaust" );
AddExhaust( th_main, 10, 2, tex );
...
\end{lstlisting}

\noindent
The texture file must be stored in DDS format in Orbiter's default texture directory. Note that \textit{oapiRegisterExhaustTexture} can be safely called multiple times with the same texture.


\subsection{Air-breathing engines}
Orbiter is not limited to rocket engines. Other devices for generating thrust can be implemented as well, from turbojet engines to solar sails or some hypothetical future technology. Unlike conventional rocket engines, which are natively supported by the Orbiter core, custom designs require a bit more work from the developer. As an example, I will here discuss the (tentative) scramjet engine implementation used by the delta-glider.\\
\\
A ramjet engine is a type of a jet engine which compresses the air for combustion not by any mechanical rotating machinery, but simply by "ramming" through the atmosphere, i.e. by using the aircraft's velocity in the airstream. This is an efficient way of generating thrust at supersonic speeds, but does not work at very low speed. (A scramjet is a variant where the air is not slowed down to subsonic speeds in the combustor and therefore avoids excessive heating at extreme velocities).\\
\\
A typical ramjet engine is composed of 3 sections:

\begin{itemize}
\item the inlet diffuser where the air is isentropically decelerated, with pressure increasing from freestream pressure $p_{\infty}$ to $p_{d}$, and temperature increasing from freestream temperature $T_{\infty}$ to $T_{d}$.
\item the combustion chamber, where the air-fuel mixture is burned at constant pressure $p_{b} = p_{d}$, and temperature increases from $T_{d}$ to $T_{b}$.
\item the exhaust nozzle, where the hot, high-pressure gas is expanded isentropically, with pressure decreasing from $p_{b}$ to $p_{\infty}$, and temperature decreasing from $T_{b}$ to $T_{e}$.
\end{itemize}

\noindent
The temperatures and pressures in the three parts of the engine (diffuser, burner and exhaust) can be calculated in the following form:

\[ T_{d} = T_{\infty}\left(1 + \frac{\gamma - 1}{2} M_{\infty}^{2}\right) \qquad p_{d} = p_{\infty}\left(\frac{T_{d}}{T_{\infty}}\right)^{\gamma / (\gamma - 1)} \]
\[ T_{b} = max(T_{b0},T_{d}) \qquad p_{b} = p_{d} \]
\[ T_{e} = T_{b}\left(\frac{p_{e}}{p_{b}}\right)^{(\gamma - 1) / \gamma} \qquad p_{e} = p_{\infty} \]

\noindent
where $M_{\infty}$ is the freestream Mach number, $\gamma$ is the ratio of specific heats, and $T_{b0}$ is the burner temperature limit, an engine design parameter defined by the heat resistance of the combustion chamber material. Note that if at high velocities $T_{d} > T_{b0}$, the engine will start to overheat purely from the isentropic compression in the diffuser, without any combustion taking place! The figure below shows an example for the temperature distribution in the engine compartments as a function of freestream Mach number. The example assumes a burner temperature limit of $T_{b0}$ = 3200 K. In this case, the limiting velocity is $v$ = Mach 8.2.

\begin{figure}[H]
  \centering
  \includegraphics[width=0.5\hsize]{ramjet_temp.png}
\end{figure}

\noindent
To calculate the thrust generated by a scramjet, we start from the fundamental thrust equation for jet propulsion,

\[ F = (\dot{m}_{a} + \dot{m}_{f})v_{e} - \dot{m}_{a} v_{\infty} + (p_{e} - p_{\infty})A_{e} \]

\noindent
where $\dot{m}_{a}$ and $\dot{m}_{f}$ are the air and fuel mass rates, respectively (using the common notation $\dot{x} = dx / dt$), $v_{e}$ and $v_{\infty}$ are the exhaust and freestream velocities, and $A_{e}$ is the exhaust cross section.\\
\\
Because of the assumption $p_{e} = p_{\infty}$ the last term vanishes. The \textit{specific thrust} is then given by

\[ \frac{F}{\dot{m}_{a}} = (1 + D)v_{e} - v_{\infty} \]

\noindent
where $D = \dot{m}_{f} / \dot{m}_{a}$ is the \textit{fuel-to-air} ratio.\\
\\
The amount of fuel burned in the combustion chamber must be adjusted so that the burner temperature limit is not exceeded. This leads to the following expression for \textit{D}:

\[ D = \frac{T_{b} - T_{d}}{Q / c_{p} - T_{b}} \]

\noindent
where \textit{Q} is a fuel-specific heating value and $c_{p}$ is the specific heat at constant pressure, given by

\[ c_{P} = \frac{\gamma R}{\gamma - 1} \]

\noindent
The mass flow of air collected by the engine is a function of air intake cross section $A_{i}$, freestream density $\rho_{\infty}$ and freestream velocity $v_{\infty}$:

\[ \dot{m}_{a} = \rho_{\infty} v_{\infty} A_{i} \]

\noindent
where $v_{\infty}$ can be expressed in terms of the freestream Mach number:

\[ v_{\infty} = M_{\infty} \sqrt{\gamma R T_{\infty}} \]

\noindent
From the above equations for \textit{D} and $\dot{m}_{a}$ we can calculate the fuel rate $\dot{m}_{f}$ required to achieve combustion temperature $T_{b}$.\\
\\
The final quantity required to calculate \textit{F} is the exhaust velocity $v_{e}$. This can be obtained from the energy balance

\[ c_{p} T_{b} = c_{p} T_{e} + v^{2}_{e} / 2 \]

\noindent
We now have all the components to calculate the thrust \textit{F} generated by the engine. The graphs below show various scramjet parameters for velocities in the range from Mach 0 to Mach 10 at an altitude of 10 km (assuming $\rho_{\infty}$ = 0.43 kg/m$^{3}$ and $T_{\infty}$ = 225 K). The DG engine design parameters in this example are \textit{Q} = 4.5 $\cdot$ 10$^{7}$ J/kg, $A_{i}$ = 0.6 m$^{2}$, and $T_{b0}$ = 3200 K.

\begin{figure}[H]
	\centering
	\subfigure{\includegraphics[width=0.49\textwidth]{fuel_to_air_ratio.png}}
	\subfigure{\includegraphics[width=0.49\textwidth]{enhaust_velocity.png}}
	\subfigure{\includegraphics[width=0.49\textwidth]{engine_thrust.png}}
	\subfigure{\includegraphics[width=0.49\textwidth]{thrust_specific_fuel_consumption.png}}
\end{figure}


\subsection{Rendering re-entry flames}
To visualise the friction heat dissipation during atmospheric reentry, Orbiter supports the rendering of "re-entry flames". To calculate the amount of heat generated per surface area and time (and to scale the exhaust flames) Orbiter uses this formula:

\[ P = \frac{1}{2} \rho v^{3} \]

\noindent
where $\rho$ is the atmospheric density, and \textit{v} is the vessel's airspeed. Orbiter renders exhaust flames if $P > P_{0}$ where $P_{0}$ is a user defined limit. The size and opacity of the reentry flames is scaled by

\[ s = min\left(1,\frac{P - P_{0}}{5P_{0}}\right) \]

\noindent
In addition, the user can specify scaling factors for length and width of the reentry texture, as well as the texture itself.\\
\\
Orbiter by default uses its own texture to render reentry flames. If you want to change the texture globally, you need to replace reentry.dds in the Textures subdirectory. If you only want to modify the texture for a specific vessel class, you need to load a custom texture, and then set your render options:

\begin{lstlisting}
void MyVessel::clbkSetClassCaps( FILEHANDLE cfg )
{
	...
	SURFHANDLE tex = oapiRegisterReentryTexture( "MyReentryFlame" );
	SetReentryTexture( tex, my_plimit, my_lscale, my_wscale );
	...
}
\end{lstlisting}

\noindent
Reentry textures require a specific layout. They consist of an elongated part in the left half of the texture map, and a circular part in the upper right corner. The lower right corner is not currently used. This is how the alpha channel of the default reentry.dds looks like:

\begin{figure}[H]
  \centering
  \includegraphics[width=0.5\hsize]{reentry_dds.png}
\end{figure}

\noindent
Note that Orbiter automatically adds a colour component to the texture depending on the value of \textit{s}, from red to white. If this is sufficient for your custom reentry flame, leave the RGB channels of the texture pure white. Otherwise you may want to experiment with additional texture colours.\\
\\
If you want to suppress rendering of reentry flames for your vessel altogether, use

\begin{lstlisting}
...
SetReentryTexture( NULL );
...
\end{lstlisting}


\subsection{Adding particle streams for exhaust and reentry effects}
Orbiter supports particle streams for rendering contrails, exhaust gases, reentry plasma trails etc. Particle streams consist of a series of textured "billboard" objects which always face the camera. The streams can be customised with a set of parameters and allow the simulation of a variety of effects.\\
\\
\textbf{The PARTICLESTREAMSPEC structure}\\
\\
At creation, the particle stream can be customised by passing a \textit{PARTICLESTREAMSPEC} structure to \textit{VESSEL::AddExhaustStream} and \textit{VESSEL::AddReentryStream}. The structure is defined as follows:

\begin{lstlisting}
typedef struct
{
	DWORD flags;
	double srcsize;
	double srcrate;
	double v0;
	double srcspread;
	double lifetime;
	double growthrate;
	double atmslowdown;
	enum LTYPE { EMISSIVE, DIFFUSE } ltype;
	enum LEVELMAP { LVL_FLAT, LVL_LIN, LVL_SQRT,
		LVL_PLIN, LVL_PSQRT } levelmap;
	double lmin, lmax;
	enum ATMSMAP { ATM_FLAT, ATM_PLIN } atmsmap;
	double amin, amax;
	SURFHANDLE tex;
} PARTICLESTREAMSPEC;
\end{lstlisting}

\noindent
\textit{srcrate}
\begin{adjustwidth}{1cm}{0cm}
The (average) rate at which particles are created by the emission source [Hz].
\\
\end{adjustwidth}


\noindent
\textit{v0}
\begin{adjustwidth}{1cm}{0cm}
The (average) emission velocity of particles by the emission source [m/s]
\\
\end{adjustwidth}

\noindent
\textit{ltype}
\begin{adjustwidth}{1cm}{0cm}
Defines the material lighting method when rendering the particles.

\begin{itemize}
\item EMISSIVE: Particles are rendered emissive (self-radiating). This is appropriate for streams representing ionized exhaust gases, or plasma streams during reentry.
\item DIFFUSE: Particles are rendered diffuse (diffuse reflection of external light sources). This is appropriate for smoke and vapour trails.
\end{itemize}
\end{adjustwidth}

\noindent
\textit{levelmap}
\begin{adjustwidth}{1cm}{0cm}
Defines the mapping between the level parameter $L$ (e.g. thruster level) and the alpha value $\alpha$ (opacity) of the generated particle. The higher the alpha value, the more solid the stream will appear. This parameter is only used for exhaust streams. The following options are available:

\begin{itemize}
\item LVL\_FLAT: constant mapping, i.e. alpha is independent of th reference level: $\alpha = lmin$
\item LVL\_LIN: linear mapping: $\alpha = L$
\item LVL\_SQRT: square root mapping: $\alpha = \sqrt{L}$
\item LVL\_PLIN: linear mapping in sub-range: $\alpha = 
\left\{
\begin{array}{ll}
	0 & L < lmin \\
	\frac{L - lmin}{lmax - lmin} & lmin \leq L \leq lmax \\
	1 & L > lmax \\
\end{array} 
\right. $
\item LVL\_PSQRT: square root mapping in sub-range: $\alpha =
\left\{
\begin{array}{ll}
	0 & L < lmin \\
	\sqrt{\frac{L - lmin}{lmax - lmin}} & lmin \leq L \leq lmax \\
	1 & L > lmax \\
\end{array} 
\right. $
\end{itemize}
\end{adjustwidth}

\noindent
\textit{lmin, lmax}
\begin{adjustwidth}{1cm}{0cm}
Defines min and max levels for alpha mapping. Only used if \textit{levelmap} is \textit{CONST}, \textit{PLIN} or \textit{PSQRT} (see above). For \textit{CONST}, only $lmin$ is used. For \textit{PLIN} and \textit{PSQRT}, $lmin < lmax$ is required. Note that $lmin < 0$ is valid - this will cause the stream to produce particles even when the reference level is 0. Likewise, $lmax > 1$ is valid - this will cause the alpha value of the particles to remain < 1 even at reference level 1.
\\
\end{adjustwidth}

\noindent
\textit{atmsmap}
\begin{adjustwidth}{1cm}{0cm}
Defines the mapping between atmospheric parameters and the alpha value $\alpha$ (opacity) of the generated particle. The following options are available:

\begin{itemize}
\item ATM\_FLAT: constant mapping, i.e. alpha is independent of atmospheric parameters: $\alpha = amin$
\item ATM\_PLIN: linear mapping of ambient atmospheric parameter $x$:\\
$\alpha = 
\left\{
\begin{array}{ll}
	0 & x < amin \\
	\frac{x - amin}{amax - amin} & amin \leq x \leq amax \\
	1 & x > amax \\
\end{array} 
\right. $
\item ATM\_PLOG: logarithmic mapping of ambient atmospheric parameter $x$:\\
$\alpha = 
\left\{
\begin{array}{ll}
	0 & x < amin \\
	\frac{ln(x / amin)}{ln(amax / amin)} & amin \leq x \leq amax \\
	1 & x > amax \\
\end{array} 
\right. $
\end{itemize}
For exhaust streams, atmospheric parameter $x$ is the ambient atmospheric density, $\rho$. For reentry streams, $x$ is defined as $x = \frac{1}{2}\rho v^{3}$ ($v$: airspeed) which is proportional to the friction power in turbulent airflow (omitting geometry-related parameters).\\
\\
\end{adjustwidth}

\noindent
\textit{amin, amax}
\begin{adjustwidth}{1cm}{0cm}
Defines min and max atmospheric parameter (ambient density or friction power) for alpha mapping. $amin < amax$ is required. For \textit{PLIN}, $amin < 0$ is admissible to enable particle generation at zero density. For \textit{PLOG}, $amin > 0$ is required. 

\begin{figure}[H]
	\centering
	\subfigure{\includegraphics[width=0.49\textwidth]{particle_level.png}}
	\subfigure{\includegraphics[width=0.49\textwidth]{particle_atmosphere.png}}
	\caption{The particle alpha value as a function of reference level (left) and atmospheric parameter (right) for different 'levelmap' and 'atmsmap' modes.}
\end{figure}
\end{adjustwidth}


\subsection{Atmospheric flight model}
\subsubsection{Lift and drag theory}
Drag is a force acting on the vessel in the direction of the freestream airflow. It is composed from several components:
\begin{enumerate}
\item The \textit{skin friction drag} caused by the boundary layer surrounding the airfoil.
\item The \textit{pressure drag} caused by separation of flow from the surface.
\item The \textit{wave drag} at supersonic velocities.
\item \textit{Induced drag}, caused by airflow around the wingtip (finite wing) from the lower to the upper surface.
\end{enumerate}
\noindent
The combination of components 1-3 is defined as \textit{profile drag} or \textit{parasite drag}.\\
\\
Lift is an upward force (perpendicular to the airflow) caused by the shape of the airfoil and its orientation to the airflow.\\
\\
Drag $D$ and lift $L$ of an airfoil are expressed by the drag and lift coefficients $c_{D}$ and $c_{L}$, with

\[ c_{D} = \frac{D}{q_{\infty} S} \]
\[ c_{L} = \frac{L}{q_{\infty} S} \]

\noindent
where $q_{\infty} = 1/2 \rho_{\infty} V^{2}_{\infty}$ is the freestream dynamic pressure, and $S$ is the wing area. Generally, $c_{D}$ and $c_{L}$, will be functions of the angle of attack, the Mach number, and the Reynolds number. We now split $c_{D}$ in the components of profile and induced drag. Induced drag is a result of lift and can be expressed as a function of $c_{L}$:

\[ c_{D} = c_{D,e} + \frac{c_{L}^{2}}{\pi eA} \]

\noindent
where $e$ is a span efficiency factor, and $A$ is the wing aspect ratio, defined as $b^{2}/S$ with wing span $b$.\\
\\
The profile component $c_{D,e}$ will change with angle of attack. We assume that $c_{D,e}$ can be expressed as the combination of a zero-lift component $c_{D,0}$ and a component depending on $c_{L}$:

\[ c_{D,e} = c_{D,0} + rc_{L}^{2} \]

\noindent
Here, $r$ is a form constant which is usually determined empirically. We can now incorporate the lift-dependent term of $c_{D,e}$ into the factor $e$, to give

\[ c_{D} = c_{D,0} + \frac{c_{L}^{2}}{\pi eA} \]

\noindent
where $\varepsilon = e / (r \pi eA + 1)$ is the \textit{Oswald efficiency factor}.\\
\\
When implementing an airfoil in Orbiter, the user must supply a function which calculates $c_{L}$ and $c_{D}$ for a given set of parameters (angle of attack, Mach number and Reynolds number). Orbiter provides a helper function (oapiGetInducedDrag) to calculate the induced drag component with the above formula.


\subsubsection{Lift and drag in transonic and supersonic flight}
(to be completed)


\subsubsection{Angular moments and vessel stability}
(to be completed)


\subsubsection{Angular drag}
Similar to (linear) drag which produces a force acting against a vessel's airspeed vector, a rotating vessel will experience angular drag which acts against the angular velocity, thus slowing the rotation. Orbiter uses the following formulae to calculate angular damping:

\[ dM_{x} = -q'S_{y}c_{\alpha,x}\omega_{x} \]
\[ dM_{y} = -q'S_{y}c_{\alpha,y}\omega_{y} \]
\[ dM_{z} = -q'S_{y}c_{\alpha,z}\omega_{z} \]

\noindent
where $q' = 1/2 \rho_{\infty}(V_{\infty} + V_{0})^{2}$ is a modified dynamic pressure which ensures that angular drag also occurs at low airspeeds (Orbiter currently uses a fixed $V_{0}$ = 30 m/s). $S_{y}$ is the vessel's cross section projected along the vertical (y) axis, used as a reference area. $S_{y}$ is the y-component of the vector passed to \textit{VESSEL::SetCrossSections()}. $c_{\alpha,x}$, $c_{\alpha,y}$ and $c_{\alpha,z}$ are the drag coefficients for rotations around the $x$, $y$, and $z$ vessel axes as defined by \textit{VESSEL::SetRotDrag()}. $\omega_{x}$, $\omega_{y}$ and $\omega_{z}$ are the angular velocities around the vessel axes, and $dM_{x}$, $dM_{y}$ and $dM_{z}$ are the changes in torque due to damping.\\
\\
Angular drag is determined by the vessel shape. Developers can adjust the effect of angular damping in the atmosphere by adjusting the coefficients passed to \textit{VESSEL::SetRotDrag()}. Higher coefficients make a vessel less responsive to control input, and reduce oscillations around equilibrium orientation.



\subsubsection{API interface for airfoil definition}
To define the lift and drag characteristics for a spacecraft in the DLL module, use the \textit{VESSEL::\-CreateAirfoil} method. An airfoil is defined as a cross section through a wing. In Orbiter, we use the term airfoil for any components of the vessel which produce lift and/or drag forces. Multiple airfoils can be defined for a single vessel (for example for the left and right wing, the body, the horizontal and vertical stabilizers in the tail, etc.). It is usually best to keep the number of airfoils low to keep the flight model predictable and to improve simulation performance.\\
\\
Orbiter distinguishes two different types of airfoil orientations: airfoils which create vertical lift (e.g. wings) and airfoils which create horizontal "lift", e.g. vertical stabilisers. Even vessels without any wings or other aerodynamic surfaces should define at least one horizontal and one vertical airfoil to define their atmospheric drag behaviour (even blunt objects such as reentry capsules which have no similarity to an aircraft produce drag and lift forces).\\
\\
When calling the \textit{CreateAirfoil} method, the user must provide
\begin{itemize}
\item basic airfoil parameters (orientation, wing area, chord length and wing aspect ratio).
\item the force attack point (i.e. the point on the vessel on which the lift and drag forces for this airfoil act). This influences the angular moments generated by the forces.
\item a callback function which calculates the lift, drag and moment coefficients of the airfoil as a function of angle of attack $\alpha$, Mach number $M$ and Reynolds number $Re$.
\end{itemize}

\noindent
The coefficients decide how much lift and drag is generated by the airfoil. The lift and drag forces ($L$ and $D$) are obtained from the moments ($c_{L}$ and $c_{D}$) by

\[ L(\alpha,M,Re) = c_{L}(\alpha,M,Re)q_{\infty}S \]
\[ D(\alpha,M,Re) = c_{D}(\alpha,M,Re)q_{\infty}S \]

\noindent
with freestream dynamic pressure $q_{\infty} = 1/2\rho v^{2}$, and reference area $S$. The function which calculates $c_{L}$ and $c_{D}$ must be able to handle arbitrary angles of attack ($-\pi$ to $\pi$) and very high Mach numbers which may occur during LEO insertion and atmospheric entry (orbital velocity for a low Earth orbit is equivalent to $M$ > 20!)\\
\\
The Reynolds number is a parameter dependent on atmospheric viscosity $\mu$:

\[ Re = \frac{\rho vc}{\mu} \]

\noindent
with freestream airspeed $v$ and density $\rho$. In the current Orbiter version, $\mu$ is assumed constant ($\mu$ = 1.6894$\cdot$10$^{-5}$ kg m-1 s-1). In future versions, $\mu$ will depend on the atmospheric composition and temperature.\\
\\
The direction of the lift force vector is defined in Orbiter as

\[ \hat{L}_{\alpha} = (0,-v_{z},v_{y}) / \sqrt{v_{y}^{2} + v_{z}^{2}} \]
\[ \hat{L}_{\beta} = (-v_{z},0,v_{x}) / \sqrt{v_{x}^{2} + v_{z}^{2}} \]

\noindent
for vertical and horizontal lift components, respectively, where ($v_{x}$,$v_{y}$,$v_{z}$) is the freestream airflow vector in vessel coordinates. This means that $\hat{L}_{\alpha}$ is rotated 90° counter-clockwise against the projection of the airflow vector into the yz-plane, and $\hat{L}_{\beta}$ is rotated 90° counter-clockwise against the projection of the airflow vector into the xz-plane. Since $\alpha$ and $\beta$ are defined as

\[ \alpha = arctan(v_{y} / -v_{z}) \]
\[ \beta = arctan(v_{x} / -v_{z}) \]

\noindent
we find the following relations between $\alpha$ or $\beta$ and the direction of lift:

\begin{table}[H]
	\centering
	\begin{tabular}{ |c|c| }
	\hline\rule{0pt}{2ex}
	\textbf{$\alpha$} & \textbf{lift direction} \\
	\hline\rule{0pt}{2ex}
	0° & up (+y)\\
	\hline\rule{0pt}{2ex}
	90° & forward (+z)\\
	\hline\rule{0pt}{2ex}
	180° & down (-y)\\
	\hline\rule{0pt}{2ex}
	270° & backward (-z)\\
	\hline
	\end{tabular}
\end{table}

\begin{table}[H]
	\centering
	\begin{tabular}{ |c|c| }
	\hline\rule{0pt}{2ex}
	\textbf{$\beta$} & \textbf{lift direction} \\
	\hline\rule{0pt}{2ex}
	0° & right (+x)\\
	\hline\rule{0pt}{2ex}
	90° & forward (+z)\\
	\hline\rule{0pt}{2ex}
	180° & left (-x)\\
	\hline\rule{0pt}{2ex}
	270° & backward (-z)\\
	\hline
	\end{tabular}
\end{table}

\noindent
This convention must be taken into account when defining the lift coefficient profile. For example, the $c_{L}$ profile for a vertical stabiliser with symmetric airfoil should be positive for 0° $\leq \beta \leq$ 90° and 180° $\leq \beta \leq$ 270°, and negative for 90° $\leq \beta \leq$ 180° and 270° $\leq \beta \leq$ 360°. The lift profile in this case may therefore resemble sin 2 $\beta$. For asymmetric airfoils the lift profile will look more complicated (for example, the zero-lift angle will usually not be exactly 0°).


\subsection{Defining an animation sequence}
 \label{ssec:def_anim_sec}
Animation sequences can be used to simulate movable parts of a vessel. Examples are the deployment of landing gear, cargo door operation, or animation of airfoils.\\
\\
Animations are implemented in \textit{vessel modules}, using the \textit{VESSEL} interface class.\\
\\
Orbiter allows 3 types of animation: rotation, translation and scaling. More complex can be built from these basic operations.


\subsubsection{Semi-automatic animation}
\textbf{Mesh requirements:}\\
\\
Animations are performed by transforming mesh groups. Therefore, all parts of the mesh participating in an animation must be defined in separate groups. Multiple groups can participate in a single transformation.\\
\\
\textbf{Defining an animation sequence:}\\
\\
Create a member function for \textit{MyVessel} to define animation sequences, and call it from the constructor, e.g.

\begin{lstlisting}
MyVessel::MyVessel( OBJHANDLE hObj, int fmodel )
: VESSEL2( hObj, fmodel )
{
	DefineAnimations();
}
\end{lstlisting}

\noindent
In the body of \textit{DefineAnimations()}, you now have to specify how the animation should be performed. Here is an example for a nose wheel animation:

\begin{lstlisting}
void MyVessel::DefineAnimations()
{
	static UINT groups[4] = {5,6,10,11};// participating groups

	static MGROUP_ROTATE nosewheel(
	0,					// mesh index
	groups, 4,			// group list and # groups
	_V( 0, -1.0, 8.5 ),	// rotation reference point
	_V( 1, 0, 0 ),		// rotation axis
	(float)(0.5 * PI)	// angular rotation range
	);

	anim_gear = CreateAnimation( 0.0 );
	AddAnimationComponent( anim_gear, 0, 1, &nosewheel );
}
\end{lstlisting}

\noindent
You first need to determine which mesh groups take part in the animation. In this case, the nose wheel consists of the four groups 5, 6, 10 and 11, and these are listed in the "groups" array.\\
\\
Next, you must define the parameters of the rotation. This is done by creating a \textit{MGROUP\_ROTATE} instance. Besides the mesh index and group indices, this also requires the rotation reference point (i.e. the point around which the mesh groups are rotated), the axis of rotation, and the rotation range.\\
\\
A new animation is created by calling \textit{CreateAnimation}. The parameter passed to \textit{CreateAnimation} defines the animation state in which the mesh groups are stored in the mesh. The return value identifies the animation.\\
\\
Finally, the rotation of the nose wheel is added to the animation by calling \textit{AddAnimationComponent}. The parameter are the animation identifier, the cutoff states of the component, and the transformation. The cutoff states define over which part of the animation the component transformation is applied. In this example, the cutoff states are 0 and 1, that is, the rotation of the nose wheel occurs over the full duration of the animation.\\
\\
Now let's consider a slightly more complicated example, where the animation consists of two components: (a) opening the wheel well cover, and (b) deploying the gear.

\begin{lstlisting}
void MyVessel::DefineAnimations()
{
	static UINT cover_groups[2] = {0,1};
	static MGROUP_ROTATE cover( 0, cover_groups, 2,
		_V( -0.5, -1.5, 7 ), _V( 0, 0, 1 ), (float)(0.45 * PI) );

	static UINT wheel_groups[4] = {5,6,10,11};
	static MGROUP_ROTATE nosewheel( 0, wheel_groups, 4,
		_V( 0, -1.0, 8.5 ), _V( 1, 0, 0 ), (float)(0.5 * PI) );

	anim_gear = CreateAnimation( 0.0 );
	AddAnimationComponent( anim_gear, 0, 0.5, &cover );
	AddAnimationComponent( anim_gear, 0.4, 1, &nosewheel );
}
\end{lstlisting}

\noindent
The rotations for the gear well cover and the landing gear are defined by two separate \textit{MGROUP\_ROTATE} variables. After creating the animation, both rotations are added as components. The cover is opened during the first part of the animation (between states 0 and 0.5) while the gear is deployed in the final part (between states 0.4 and 1). Note that there is a small overlap (between 0.4 and 0.5), which means that the gear begins to rotate before the cover is fully opened.\\
\\
When the animation is played backward to retract the gear, the components are rotated in the inverse order: the gear is retracted first, then the cover is closed.\\
\\
Animations can be arranged in a hierarchical order, so that a parent animation can transform mesh groups which are themselves animations. Consider for example the wheel on a landing gear which is spinning while the gear is being retracted. In this case, the gear animation is defined as a rotation around the gear hinge point, while the wheel animation is a rotation around the wheel axis. The wheel animation must be defined as a child of the gear animation, because the wheel is rotated together with the gear.

\begin{lstlisting}
void MyVessel::DefineAnimations()
{
	ANIMATIONCOMPONENT_HANDLE parent;

	static UINT gear_groups[2] = {5,6};
	static MGROUP_ROTATE gear( 0, gear_groups, 2,
	_V( 0, -1.0, 8.5 ), _V( 1, 0, 0 ), (float)(0.45 * PI) );

	static UINT wheel_groups[2] = {10,11};
	wheel = new MGROUP_ROTATE( 0, wheel_groups, 2,
	_V( 0, -1.0, 6.5 ), _V( 1, 0, 0 ), (float)(2 * PI) );

	anim_gear = CreateAnimation( 0.0 );
	parent = AddAnimationComponent( anim_gear, 0, 1, &gear );

	anim_wheel = CreateAnimation( 0.0 );
	AddAnimationComponent( anim_wheel, 0, 1, wheel, parent );
}
\end{lstlisting}

\noindent
The gear and wheel rotations are defined by the \textit{MGROUP\_ROTATE} variables "gear" and "wheel". Note that in this case "wheel" is not defined static, since reference point and axis will be modified by the parent. Therefore, "wheel" must be defined as a data member of the \textit{MyVessel} class. Since "wheel" is allocated dynamically, don't forget to de-allocate it with

\begin{lstlisting}
MyVessel::~MyVessel()
{
	...
	delete wheel;
	...
}
\end{lstlisting}

\noindent
The return value of the \textit{AddAnimationComponent()} call for the gear animation is a handle which identifies the transformation. We use this value for the optional parent parameter when defining the animation component for the wheel animation. This makes the wheel animation a child of the gear animation.\\
\\
A complex example for hierarchical animations can be found in the RMS arm animation of Space Shuttle Atlantis in \textit{Orbitersdk\textbackslash samples\textbackslash Atlantis\textbackslash Atlantis.cpp}.\\
\\
Apart from rotations, mesh groups can also be transformed by translating and scaling. The corresponding \textit{MGROUP\_TRANSFORM} derivates are \textit{MGROUP\_TRANSLATE} and \textit{MGROUP\_SCALE}:

\begin{lstlisting}
	MGROUP_TRANSLATE t1( 0, groups, 2, _V( 0, 10, 5 ) );
	MGROUP_SCALE t2( 0, groups, 2, _V( 5, 0, 2 ), _V( 2, 2, 2 ) );
\end{lstlisting}

\noindent
In both cases, the first three parameters are the same as for \textit{MGROUP\_ROTATE} (mesh, index, group list and number of groups). For \textit{MGROUP\_TRANSLATE}, the last parameter defines the translation vector. For \textit{MGROUP\_SCALE}, the last two parameters define the scale origin, and the scale factors in the three axes.\\
\\
\textbf{Performing the animation:}\\
\\
To animate the nose wheel now, we need to manipulate the animation sequence state by calling \textit{SetAnimation()} with a value between 0 (fully retracted) and 1 (fully deployed). This is typically done in the \textit{Timestep()} member function, e.g.

\begin{lstlisting}
void MyVessel::Timestep( double simt )
{
	if (gear_status == CLOSING || gear_status == OPENING)
	{
		double da = oapiGetSimStep() * gear_speed;
		if (gear_status == CLOSING)
		{
			if (gear_proc > 0.0)
				gear_proc = max(0.0, gear_proc - da);
			else
				gear_status = CLOSED;
		}
		else// door opening
		{
			if (gear_proc < 1.0)
				gear_proc = min(1.0, gear_proc + da);
			else
				gear_status = OPEN;
		}
		SetAnimation( anim_gear, gear_proc );
	}
}
\end{lstlisting}

\noindent
Here, gear\_status is a flag defining the current operation mode (\textit{CLOSING}, \textit{OPENING}, \textit{CLOSED}, \textit{OPEN}). This will typically be set by user interaction, e.g. by pressing a keyboard button. If the animation is in progress (\textit{OPENING} or \textit{CLOSING}), we determine the rotation step (da) as a function of the current frame interval (\textit{oapiGetTimeStep}). The value of gear\_speed defines how fast the gear is deployed.\\
\\
Next, we update the deployment state (\textit{gear\_proc}), and check whether the sequence is complete ($leq 0$ if closing, or $\geq 1$ if opening). Finally, \textit{SetAnimation} is called to perform the animation.\\
\\
The DeltaGlider sample module (\textit{Orbitersdk\textbackslash samples\textbackslash DeltaGlider}) contains a complete example for an animation implementation.


\subsubsection{Manual animation}
As an alternative to the (semi-)automatic animation concept described in the previous section, Orbiter also allows manual animation. This can be more versatile, but requires more effort from the module developer, because the complete animation sequence must be implemented explicitly.\\
\\
A manual animation sequence is created by the functions \textit{VESSEL::RegisterAnimation()} and \textit{VESSEL::UnregisterAnimation()}. A call to \textit{RegisterAnimation} causes Orbiter to call the module's \textit{ovcAnimate} callback function at each frame, provided the vessel's visual exists. \textit{UnregisterAnimation} cancels the request.\\
\\
Note that \textit{RegisterAnimation/UnregisterAnimation} pairs can be nested. Each call to \textit{RegisterAnimation} increments a reference counter, each call to \textit{UnregisterAnimation} decrements the counter. Orbiter will call \textit{ovcAnimate} as long as the counter is > 0.\\
\\
It is up to the module to implement its animations in the body of \textit{ovcAnimate}. Typically this will involve calls to \textit{MeshgroupTransform()}, to rotate, translate or scale mesh groups as a function of the last simulation time step. Note that \textit{ovcAnimate} is called only once per frame, even if more than one \textit{RegisterAnimation} request has been logged. The module must therefore decide which animations need to be processed in \textit{ovcAnimate}.\\
\\
\textit{UnregisterAnimation} should never be called from inside \textit{ovcAnimate}, since \textit{ovcAnimate} is only called if the visual exists. This could cause the unregister request to be lost. It is better to test for animation termination in \textit{ovcTimestep}.


\subsection{Designing 2D-instrument panels}
\label{ssec:2d_panel_design}
Instrument panels are a good way to give an individual feel to a spacecraft class and allow the user to monitor flight parameters and control specific aspects of the vessel via the mouse, without the need to remember a large number of keyboard commands.\\
\\
There are two ways to define a cockpit interior: you can build one (or several) flat two-dimensional panels as bitmaps which are overlayed on top of the three-dimensional scenery of the simulation window (denoted as \textit{panels} below), or you can construct a full three-dimensional mesh representation of the cockpit (denoted as \textit{virtual cockpit}, or \textit{VC} below). A vessel can implement both 2-D panels and virtual cockpits. The user can switch between them (and the generic cockpit view comprising two MFD displays and HUD) by pressing \keystroke{F8}.\\
\\
In this section we will discuss the steps required to define 2-D panels in the vessel module. Section \ref{ssec:vc_design} will deal with virtual cockpits.

\subsubsection{The panel request callback function}
Whenever the vessel switches to a new 2-D panel cockpit view (either from an outside view or another cockpit view), it calls the \textit{clbkLoadPanel2D} callback function. This is the point where we need to define the panel geometry and functions. For now, we are going to implement only a single main panel:

\begin{lstlisting}
bool MyVessel::clbkLoadPanel2D( int id, PANELHANDLE hPanel,
	DWORD viewW, DWORD viewH )
{
	switch (id)
	{
		case 0: 
			DefineMainPanel( hPanel );
			ScalePanel( hPanel, viewW, viewH );
			return true;
		default:
			return false;
	}
}
\end{lstlisting}

\noindent
Note that \textit{clbkLoadPanel2D} has been introduced in the \textit{VESSEL3} interface, so your vessel class must be derived from \textit{VESSEL3} to make use of it. \textit{clbkLoadPanel2D} is the equivalent of \textit{clbkLoadPanel} for the old-style 2-D panel interface. If your vessel defines \textit{clbkLoadPanel2D}, it should \textit{not} also define \textit{clbkLoadPanel}.\\
\\
The \textit{id} parameter defines the panel (the main panel has always \textit{id} 0, but additional neighbour panels can be defined as well). The \textit{hPanel} object is a handle that is required by various functions during the definition of the panel. The \textit{viewW} and \textit{viewH} parameters define the width and height of the viewport in pixels, which can be useful for scaling purposes.\\
\\
Now we need to implement the \textit{DefineMainPanel} function which defines the panel mesh, textures and active areas.


\subsubsection{The panel mesh}
2-D instrument panels are defined as \textit{2-D meshes}. Orbiter uses the same mesh format for 2-D meshes as it does for 3-D meshes (used e.g. to describe vessel and virtual cockpit geometries), with the exception that the vertex z-coordinates for 2-D meshes are ignored and should be set to 0.\\
\\
The mesh coordinate system to which the mesh vertex coordinates refer can be freely chosen by the developer. A convenient convention is to set the bottom left corner of the mesh to coordinates (0,0), and the top right corner to coordinates (\textit{px},\textit{py}), where \textit{px} and \textit{py} are the width and height of the panel background texture in pixels. With this convention, mesh coordinates correspond to the pixel positions of the background texture.\\
\\
As a first example, let's start with a simple rectangular panel, which can be defined with 4 vertices and 2 triangles. If we plan for a panel texture of dimension 1280x400, then the mesh would look like this:
\begin{itemize}
\item Vertex coordinate list: (0,0,0), (0,400,0), (1280,400,0), (1280,0,0)
\item Triangle index list: (0,2,1), (2,0,3)
\end{itemize}

\noindent
In principle it is possible to put this mesh definition into a standard Orbiter mesh file, and read it when required with \textit{oapiLoadMesh}. However, this mesh is so simple that it is more efficient to define it directly in the vessel code. Defining the 2-D panel mesh in the code will later also have the advantage that we are better able to control animations and moving parts which require direct access to the vertex lists. The main panel mesh definition could look like this:

\begin{lstlisting}
void MyVessel::DefineMainPanel( PANELHANDLE hPanel )
{
	static DWORD panelW = 1280;
	static DWORD panelH = 400;
	float fpanelW = (float)panelW;
	float fpanelH = (float)panelH;
	static NTVERTEX VTX[4] = {
		{      0,       0, 0, 0, 0, 0, 0, 0},
		{      0, fpanelH, 0, 0, 0, 0, 0, 0},
		{fpanelW, fpanelH, 0, 0, 0, 0, 0, 0},
		{fpanelW,       0, 0, 0, 0, 0, 0, 0}
		};
	static WORD IDX[6] = {
		0, 2, 1,
		2, 0, 3
		};

	if (hPanelMesh) oapiDeleteMesh( hPanelMesh );
	hPanelMesh = oapiCreateMesh( 0, 0 );
	MESHGROUP grp = {VTX, IDX, 4, 6, 0, 0, 0, 0, 0};
	oapiAddMeshGroup( hPanelMesh, &grp );
	SetPanelBackground( hPanel, 0, 0, hPanelMesh, panelW, panelH, 0,
		PANEL_ATTACH_BOTTOM | PANEL_MOVEOUT_BOTTOM );
}
\end{lstlisting}

\noindent
Here, \textit{hPanelMesh} is assumed to be a \textit{MESHHANDLE} object defined as a member of \textit{MyVessel}. The call to \textit{oapiCreateMesh} creates an empty mesh, to which the group for the main panel background is added by \textit{oapiMeshGroup}.\\
\\
Since the \textit{hPanelMesh} object may be shared with other cockpit panel views, we need to check if it is allocated already, and delete it before defining the new one, using the \textit{oapiDeleteMesh} function. For this to work, it must be initialised it to \textit{NULL} in the constructor:

\begin{lstlisting}
MyVessel::MyVessel( OBJHANDLE hObj, int fmodel )
{
	...
	hPanelMesh = NULL;
	...
}
\end{lstlisting}

\noindent
To avoid memory leaks, the destructor should delete the mesh if required:

\begin{lstlisting}
MyVessel::~MyVessel()
{
	...
	if (hPanelMesh) oapiDeleteMesh( hPanelMesh );
	...
}
\end{lstlisting}

\noindent
The \textit{SetPanelBackground} call in our \textit{DefineMainPanel} function registers the panel mesh with Orbiter. Its parameters are:

\begin{itemize}
\item The panel handle, as provided by \textit{clbkLoadPanel2D}
\item A list of textures, and the number of textures in the list (set to 0 for now - we'll come back to those later)
\item The panel mesh handle
\item The width and height of the panel in mesh units
\item The panel base line
\item The viewport attachment and scroll flags
\end{itemize}


\subsubsection{Scaling the panel}
Now we have to think about scaling the panel to the viewport. This will be done in the \textit{ScalePanel} method that has already been called in \textit{clbkLoadPanel2D}.\\
\\
By default, we want to scale the panel so that it fills the width of the viewport, independent of its actual size. This can be done painlessly by using the \textit{VESSEL3::SetPanelScaling} method. This is a big improvement over the old-style panel definitions, which only provided an awkward global scaling option. In addition, we can also define a zoom option that magnifies the panel. This will only display a part of the panel, but other parts can be scrolled in. This is particularly useful for small viewport sizes, where scaling the panel to fit would make it too small to use. The user can switch between standard and magnified scaling with the mouse wheel.\\
\\
The scaling parameters passed to \textit{SetPanelScaling} are magnification factors that describe how many viewport pixels should be covered by one mesh unit. The implementation of \textit{ScalePanel} could therefore look like this:

\begin{lstlisting}
void MyVessel::ScalePanel( PANELHANDLE hPanel, DWORD viewW, DWORD viewH )
{
	double defscale = (double)viewW / 1280.0;
	double magscale = max(defscale, 1.0);
	SetPanelScaling( hPanel, defscale, magscale );
}
\end{lstlisting}

\noindent
The \textit{defscale} factor makes sure that the panel (defined as size 1280) stretches over the full viewport width (viewW). The \textit{magscale} factor magnifies the panel such that 1 mesh unit covers one screen pixel if the viewport width is smaller than the panel width. This is a sensible convention, but of course you are free to implement different scaling strategies for your panels.


\subsubsection{Adding a panel background texture}
We now want to draw a texture over the bare panel mesh. The texture serves the same function as the bitmap in the old-style panel definitions, but it must be stored in DDS format, rather than BMP format. You may have to experiment with the compression format, but usually DXT1 is best if no or only binary transparency is required, or DXT5 if continuous transparency is required.\\
\\
Another important restriction is the fact that textures must have sizes that are multiples of 2. So for our 1280x400 texture we will have to create a 2048x512 pixel texture. For now this is a lot of waste, but we can use the same texture to add additional panels and active elements later on. Sometimes you may also be able to reduce the required texture size by clever mesh design and re-using the same texture elements multiple times (e.g. defining the right half of the panel as a mirror of the left).\\
\\
The panel texture is a global resource (it is shared by all vessels of the MyVessel class, so we can make it static and load it during module initialisation:

\begin{lstlisting}
// vessel class interface
class MyVessel: public VESSEL3
{
	public:
		...
		static SURFHANDLE panel2dtex;
		...
};

// public member initialisation
SURFHANDLE MyVessel::panel2dtex = NULL;

// module initialisation
DLLCLBK void InitModule( HINSTANCE hModule )
{
	...
	MyVessel::panel2dtex = oapiLoadTexture( "MyVessel\\panel2d.dds" );
	...
}

// module cleanup
DLLCLBK void ExitModule( HINSTANCE hModule )
{
	...
	oapiDestroySurface( MyVessel::panel2dtex );
	...
}
\end{lstlisting}

\noindent
where the panel texture is assumed to be located in file \textit{Textures\textbackslash MyVessel\textbackslash panel2d.dds}.\\
\\
We can now modify the \textit{DefineMainPanel} method to make use of the background texture:

\begin{lstlisting}
void MyVessel::DefineMainPanel( PANELHANDLE hPanel )
{
	static DWORD panelW = 1280;
	static DWORD panelH = 400;
	float fpanelW = (float)panelW;
	float fpanelH = (float)panelH;
	static DWORD texW = 2048;
	static DWORD texH = 512;
	float ftexW = (float)texW;
	float ftexH = (float)texH;
	static NTVERTEX VTX[4] = {
		{      0,       0, 0, 0, 0, 0,
			           0.0f, 1.0f - fpanelH / ftexH},
		{      0, fpanelH, 0, 0, 0, 0,
			           0.0f, 1.0f                  },
		{fpanelW, fpanelH, 0, 0, 0, 0,
			fpanelW / ftexW, 1.0f                  },
		{fpanelW,       0, 0, 0, 0, 0,
			fpanelW / ftexW, 1.0f - fpanelH / ftexH}
		};
	static WORD IDX[6] = {
		0, 2, 1,
		2, 0, 3
		};

	if (hPanelMesh) oapiDeleteMesh( hPanelMesh );
	hPanelMesh = oapiCreateMesh( 0, 0 );
	MESHGROUP grp = {VTX, IDX, 4, 6, 0, 0, 0, 0, 0};
	oapiAddMeshGroup( hPanelMesh, &grp );
	SetPanelBackground( hPanel, &panel2dtex, 1, hPanelMesh, panelW, panelH, 0,
		PANEL_ATTACH_BOTTOM | PANEL_MOVEOUT_BOTTOM );
}
\end{lstlisting}

\noindent
The texture coordinates for the mesh vertices have now been defined (where I am assuming that the main panel image is located in the lower left corner of the texture. The call to \textit{SetPanelBackground} contains a pointer to the texture handle, and the number of textures (1). If your panel mesh references more than one texture, put them in a list, pass the list as the second parameter of \textit{SetPanelBackground}, and the number of textures in the list as the third parameter.\\
\\
At this point, you can compile your vessel code and run it in Orbiter. It isn't very exciting yet (a static panel background texture covering the lower half of the screen), but it is the basis for the next steps. You should be able to scroll the panel up and down with the cursor keys.


\subsection{Designing instrument panels (legacy style)}
This section describes the design for a legacy-style 2-D instrument panel. This method is still supported by Orbiter, but its use is discouraged, because it does not work well with newer 3-D rendering engines and external graphics clients. Vessel addon designers should switch to the new 2-D panel method described in section \ref{ssec:2d_panel_design}.

\subsubsection{Defining a panel}
You will first need to create a bitmap which represents the 2-D instrument panel. You can use any paint tool capable of generating Windows BMP files. The panel can be saved in 8-bit or 24-bit mode, but 8-bit mode is strongly recommended to reduce the size of the resulting vessel module, and improve simulation performance.

\begin{figure}[H]
	\centering
	\subfigure{\includegraphics[width=0.99\textwidth]{dg_panel.png}}
	\caption{The DG main panel bitmap.}
\end{figure}

\noindent
Some thought should be given to the size of the panel bitmap. Remember that users will run Orbiter at different screen resolutions and window sizes. If the bitmap is made very large, a lot of panning will be required to bring different parts of the panel into view at low resolutions. If the bitmap is very small, it will cover only a small area of the screen at high resolutions. It is probably best to design panels for medium screen resolutions (between 1024x768 and 1280x960 pixels). Users with very low or very high screen resolutions will be able to adjust the panel size by using Orbiter's panel rescaling option.\\
\\
You should also consider whether the panel is to cover the whole screen, or only part of it. The main panel should usually obstruct only part of the 3-D scenery, but side panels could take up the whole simulation window.\\
\\
The main panel should typically also provide space for MFDs (multifunctional displays), which are the primary method to provide flight data to the pilot. Most common is a layout with two MFDs, but fewer or more can be defined as well. The size of the MFD displays should be chosen so that they are easily readable over a 'typical' range of screen resolutions.\\
\\
You can define more than one panel for a vessel. For example, you may define a main panel which is visible in the lower half of the screen when the pilot looks forward, an overhead panel, side panels, etc. The user can switch between the different panels with \Ctrl\DArrow\UArrow\RArrow\LArrow keys. We will discuss later how to define the connectivity between panels. To start with, let's look at the definition of a single main panel.\\
\\
Once you have created the panel BMP file, you should add it as a bitmap resource to your vessel module project. Now you are ready to write the code to support the panel. To do so, you need to overload the \textit{clbkLoadPanel} method of the \textit{VESSEL2} class:

\begin{lstlisting}
bool MyVessel::clbkLoadPanel( int id )
{
	...
}
\end{lstlisting}

\noindent
Here we assume that \textit{MyVessel} is a class derived from \textit{VESSEL2} (see section \ref{ssec:vessel_init} on how to create vessel instances). \textit{id} is a panel identifier which Orbiter will provide to let your function know which panel is required. If only a single main panel is defined, \textit{id} will always be 0. If you define more than one panel, you should examine this parameter to decide which panel to load.\\
\\
Orbiter will call your \textit{clbkLoadPanel} method whenever it needs to load an instrument panel. This happens if

\begin{itemize}
\item the user switches to instrument panel view from another view mode by pressing \keystroke{F8}.
\item the user switches between panels with \Ctrl\DArrow\UArrow\RArrow\LArrow keys.
\item the user switches from an external view to a cockpit view.
\item the user switches to a different spacecraft with \keystroke{F3}.
\end{itemize}

\noindent
In the body of \textit{clbkLoadPanel}, we need to load the panel bitmap and pass it to Orbiter via the \textit{oapiRegisterPanelBackground} function:

\begin{lstlisting}
bool MyVessel::clbkLoadPanel( int id )
{
	HBITMAP hBmp = LoadBitmap( hDLL, MAKEINTRESOURCE(IDB_PANEL) );
	oapiRegisterPanelBackground( hBmp );
	return true;
}
\end{lstlisting}

\noindent
Here, \textit{hDLL} is a module instance handle passed to the \textit{InitModule} callback function of your module, and \textit{IDB\_PANEL} is assumed to be the numerical resource identifier of the panel bitmap. The return value of \textit{clbkLoadPanel} should normally be \textit{true}. \textit{false} signifies an error, e.g. failure to load the panel bitmap.\\
\\
\textit{oapiRegisterPanelBackground} has an additional optional parameter which defines how the panel is connected to the edges of the simulation window, and how it can be scrolled across the screen with the cursor keys. A common choice for a main window is to connect it to the lower edge of the window, and allow it to be scrolled downward. This can be accomplished as follows:

\begin{lstlisting}
	oapiRegisterPanelBackground( hBmp,
		PANEL_ATTACH_BOTTOM|PANEL_MOVEOUT_BOTTOM );
\end{lstlisting}

\noindent
(This is in fact the default setting, so you only need to provide this parameter if you need to define a different behaviour.) For a full list of supported attachment and scroll parameters, see the \textit{oapiRegisterPanelBackground} description in the Reference Manual.\\
\\
\textit{oapiRegisterPanelBackground} has a further optional parameter to define a transparent colour. Any part of the bitmap containing that colour will be transparent in the render window. This allows to implement irregular panel shapes such as windows which provide a view of the 3-D scene though the panel.\\
\\
The transparent colour is given in 0xRRGGBB format. Note that if Orbiter is run in 16-bit mode, not all colours can be represented. For that reason, it is recommended to use either black (0x000000) or white (0xFFFFFF) as the transparent colour which are always available in 16-bit mode, to avoid problems. In any case, you should always check that your panel appears correctly in both 16 and 32 bit modes before publishing your addon.\\
\\
So the final version of our main panel loading call looks like this, where we allow the panel to be scrolled out at the bottom, and use white as the transparent colour:

\begin{lstlisting}
	oapiRegisterPanelBackground( hBmp,
		PANEL_ATTACH_BOTTOM|PANEL_MOVEOUT_BOTTOM, 0xFFFFFF );
\end{lstlisting}

\noindent
At this point, you can try to compile your module and test the panel in Orbiter. You should be able to make the panel visible by pressing \keystroke{F8} when you are in the cockpit of an instance of your vessel class, and scroll it up and down with the cursor keys.


\subsubsection{Defining active panel areas}
Now we can start to do something interesting with our new panel. We need to activate areas of the panel. Active areas can do two things:

\begin{itemize}
\item They can be repainted from within the code, for example to dynamically update an instrument display, and/or
\item they can register mouse button events to allow the user to interact with the panel.
\end{itemize}

\noindent
A panel area is activated with the \textit{oapiRegisterPanelArea} function. This must be called in your vessel's \textit{clbkLoadPanel} method, after the panel has been loaded with \textit{oapiRegisterPanelBackground}. Let's define an area that contains a button which the user can press:

\begin{lstlisting}
	oapiRegisterPanelArea( AID_BUTTON, _R( 10, 10, 30, 20 ),
		PANEL_REDRAW_MOUSE, PANEL_MOUSE_LBDOWN, PANEL_MAP_BACKGROUND );
\end{lstlisting}

\noindent
The first parameter, \textit{AID\_BUTTON}, is a value that uniquely identifies the area across all panels. The next parameter defines a rectangular area in the panel given by the left, top, right and bottom edges (measured from the top left corner of the panel bitmap).\\
\\
The next parameter, \textit{PANEL\_REDRAW\_MOUSE}, specifies that the area must be redrawn whenever a mouse event occurs inside the area. Other areas may need to be redrawn at each frame, by explicitly requesting a redraw, or not at all.\\
\\
\textit{PANEL\_MOUSE\_LBDOWN} requests a notification whenever the user presses the left mouse button inside the area. You can also request mouse button releases, or continuous notifications as long as a button is pressed. A panel area defined with \textit{PANEL\_MOUSE\_IGNORE} will never generate any mouse events.\\
\\
\textit{PANEL\_MAP\_BACKGROUND} requests the area background (i.e. the portion of the panel bitmap under the area) to be passed to the redraw function. Instead, you could request the current status of the area, or an un-initialised bitmap to be passed to the redraw function. See the documentation to \textit{oapiRegisterPanelArea} in the Reference Manual for more details.\\
\\
You can define more panel areas to turn your panel into a useful interface, but avoid overlapping areas.\\
\\
Next, we need to implement the callback functions Orbiter will call to allow the module to respond to redraw and mouse events generated by the active areas.


\subsubsection{The mouse event handler}
To intercept mouse events generated by a panel you must overload the \textit{clbkPanelMouseEvent} method of the \textit{VESSEL2} class:

\begin{lstlisting}
bool MyVessel::clbkPanelMouseEvent( int id, int event, int mx, int my )
{
	...
}
\end{lstlisting}

\noindent
where \textit{id} is the identifier of the panel area for which the event was generated (e.g. \textit{AID\_BUTTON} in our example), \textit{event} specifies the mouse event type, and \textit{mx},\textit{my} are the panel coordinates at which the event occurred.\\
\\
\textbf{Important}: A button-up event is always generated for the instrument which produced the preceding button-down event, even if the mouse has been dragged out of the panel area in the mean time.\\
\\
The following mouse events are available:\\
\textit{PANEL\_MOUSE\_LBDOWN} - Left mouse button pressed down.\\
\textit{PANEL\_MOUSE\_RBDOWN} - Right mouse button pressed down.\\
\textit{PANEL\_MOUSE\_LBUP} - Left mouse button released.\\
\textit{PANEL\_MOUSE\_RBUP} - Right mouse button released.\\
\textit{PANEL\_MOUSE\_LBPRESSED} - Left mouse button down\\
\textit{PANEL\_MOUSE\_RBPRESSED} - Right mouse button down.\\
\\
The \textit{PANEL\_MOUSE\_LBPRESSED} and \textit{PANEL\_MOUSE\_RBPRESSED} events are sent continuously while the buttons are held down. This allows the implementation of mouse-dragging event, for example to move sliders with the mouse.\\
\\
Inside \textit{clbkPanelMouseEvent}, your code must check the area id and perform the appropriate actions:

\begin{lstlisting}
bool MyVessel::clbkPanelMouseEvent( int id, int event, int mx, int my )
{
	switch (id)
	{
		case AID_BUTTON:
			DoProcessButtonPress(...);
			return true;
		case ... // place response to other areas here
	}
	return false;
}
\end{lstlisting}

\noindent
Here, \textit{DoProcesButtonPress} is assumed to be a locally defined method which performs the required action.\\
\\
The return value is currently only used for areas which use the \textit{PANEL\_REDRAW\_MOUSE} flag. In this case, returning \textit{true} will trigger a redraw event, while returning \textit{false} will not. For efficiency, return \textit{true} only if the area needs to be redrawn as a consequence of the mouse event.\\
\\
The \textit{mx} and \textit{my} parameters define the area coordinates (0,0 is the top left corner of the area) at which the mouse event occurred. This is sometimes useful to fine-tune the response. For example, let's assume that the button defined in the example is actually a switch which can be flipped left or right. Then we could do this:

\begin{lstlisting}
	...
	case AID_BUTTON:
		if (mx < 10)
			DoProcessFlipLeft(...);
		else
			DoProcessFlipRight(...);
		return true;
	...
\end{lstlisting}


\subsubsection{The redraw event handler}
To provide a visual cue of the button press, we may want to redraw the area (e.g. to simulate a control lamp lighting up). Other areas representing gauges and displays may need to be redrawn continuously without any mouse events. To respond to redraw requests, we need to overload the \textit{clbkPanelRedrawEvent} method of the \textit{VESSEL2} class:

\begin{lstlisting}
bool MyVessel::clbkPanelRedrawEvent( int id, int event, SURFHANDLE surf )
{
	...
}
\end{lstlisting}

\noindent
As with the mouse event handler, your implementation of \textit{clbkPanelRedrawEvent} should examine the area \textit{id} (and the redraw event, if required), and redraw the corresponding area as required.\\
\\
\textit{surf} is a handle to the paint surface for the area in which all repainting takes place. The contents of the surface passed to the callback function depend on the parameters specified during the definition of the area:\\
\textit{PANEL\_MAP\_NONE} - surf is undefined\\
\textit{PANEL\_MAP\_BACKGROUND} - surf contains area background\\
\textit{PANEL\_MAP\_CURRENT} - surf contains current area contents\\
\textit{PANEL\_MAP\_BGONREQUEST} - surf is undefined, but area background can be obtained on request\\
\\
\textit{PANEL\_MAP\_NONE} is the most efficient option if the whole area needs to be redrawn at each redraw event. \textit{PANEL\_MAP\_BACKGROUND} is least efficient, because it involves the most internal surface copy processes. If you need the background bitmap, but your area doesn't need to be redrawn for each redraw request generated (for example, if you have defined a gauge, to be redrawn at each simulation frame, but often the contents don't change between subsequent frames), it is more efficient to use the \textit{PANEL\_MAP\_BGONREQUEST} flag, and obtaining the background bitmap explicitly with a call to \textit{oapiBltPanelAreaBackground} whenever the area actually needs to be redrawn (see documentation to \textit{oapiBltPanelAreaBackground} in the Reference Manual for more details).\\
\\
Our redraw function might look like this:

\begin{lstlisting}
bool MyVessel::clbkPanelRedrawEvent( int id, int event, SURFHANDLE surf )
{
	switch (id)
	{
		case AID_BUTTON:
			if (button_pressed)
				oapiBlt( surf, buttonSurf, 0, 0, 0, 0, 20, 10 );
			else
				oapiBlt( surf, buttonSurf, 0, 0, 0, 10, 20, 10 );
			return true;
		case ... // imprement redraw methods for other areas
	}
	return false;
}
\end{lstlisting}

\noindent
Here, \textit{buttonSurf} is assumed to be the surface handle to a bitmap which contains images of the button for both the pressed and the released state. (You can store this bitmap as a module resource and obtain a surface handle to it with the \textit{oapiCreateSurface} method.) \textit{oapiBlt} copies the relevant part of the bitmap into the area's surface (the \textit{button\_pressed} flag could for example have been set in the mouse event handler).\\
\\
When more complex redrawing is required, you can obtain a device context handle to the surface with \textit{oapiGetDC} and then use standard Windows GDI methods to paint in the surface. (see Windows API documentation). Don't forget to release the device context with \textit{oapiReleaseDC} at the end.\\
\\
The return value of \textit{clbkPanelRedrawEvent} signals to Orbiter if the contents of the area have been redrawn. Return \textit{true} only if you did modify the surface, \textit{false} otherwise.


\subsubsection{Defining panel MFDs}
\label{sssec:def_panel_mfd}
MFD (multifunctional displays) are probably the most important components of your panel. They are defined differently to other panel areas, because some of the redraw events are processed directly by Orbiter.\\
\\
MFDs consist of a square display area (representing a colour CRT or LCD display) and rows of control buttons to the left and right. The number of buttons can be defined individually.\\
\\
You reserve a panel area for an MFD with the \textit{oapiRegisterMFD} method during setting up the panel in the overloaded \textit{clbkLoadPanel} callback function:

\begin{lstlisting}
bool MyVessel::clbkLoadPanel( int id )
{
	oapiRegisterPanelBackground(...);
	...
	MFDSPEC mfds_left  = {{100, 10, 200, 110}, 6, 6, 10, 20};
	oapiRegisterMFD( MFD_LEFT,  mfds_left );
	...
	return true;
}
\end{lstlisting}

\noindent
The first parameter of \textit{oapiRegisterMFD} identifies the MFD (left, right, or a user-defined MFD). The left and right MFDs can be controlled with keyboard commands, while user-defined MFDs can only be controlled with the mouse. Therefore you should always first define the left and right MFDs, and use user-defined ones only if more than two MFDs are to be defined in the panel.\\
\\
The second parameter is a structure which defines the layout of the MFD. It contains:

\begin{itemize}
\item the rectangular area (left, top, right and bottom edge) of the panel area to contain the MFD display,
\item the number of buttons along the left and right edges,
\item the y-offset of the upper edge of the topmost button from the top edge of the display,
\item the y-distance between the top edges of the buttons.
\end{itemize}

\noindent
The button rows must be implemented as separate areas. Note that a single area is used for the left row of buttons, and another one for the right row. In addition, a bottom row of 3 buttons can be defined to perform MFD on/off, display of button commands, and display of mode contents:

\begin{lstlisting}
bool MyVessel::clbkLoadPanel( int id )
{
	oapiRegisterPanelBackground(...);
	...
	MFDSPEC mfds_left  = {{100, 10, 200, 110}, 6, 6, 10, 20};
	oapiRegisterMFD( MFD_LEFT,  mfds_left );
	oapiRegisterPanelArea(
		AID_LBUTTONS, _R( 80, 20, 100, 100 ), PANEL_REDRAW_USER,
		PANEL_MOUSE_LBDOWN | PANEL_MOUSE_LBPRESSED, PANEL_MAP_BACKGROUND );
	oapiRegisterPanelArea(
		AID_RBUTTONS, _R( 200, 20, 220, 100 ), PANEL_REDRAW_USER,
		PANEL_MOUSE_LBDOWN | PANEL_MOUSE_LBPRESSED, PANEL_MAP_BACKGROUND );
	oapiRegisterPanelArea(
		AID_BBUTTONS, _R( 100, 110, 200, 130 ), PANEL_REDRAW_NEVER,
		PANEL_MOUSE_LBDOWN );
	...
	return true;
}
\end{lstlisting}

\noindent
The button areas have been defined with the \textit{PANEL\_MOUSE\_LBPRESSED} flag in addition to \textit{PANEL\-\_MOUSE\_LBDOWN}, so that continued mouse presses can be recorded when required.\\
\\
Mouse button events now need to be processed in the mouse event handler:

\begin{lstlisting}
bool MyVessel::clbkPanelMouseEvent( int id, int event, int mx, int my )
{
	switch (id)
	{
		case AID_LBUTTONS:
		case AID_RBUTTONS:
			if (my % 20 < 15)
			{
				int bt = my / 20 + (id == AID_LBUTTONS ? 0 : 6);
				oapiProcessMFDButton( MFD_LEFT, bt, event );
				return true;
			}
			break;
		case ...
	}
	return false;
}
\end{lstlisting}

\noindent
This code fragment processes all the buttons in the left and right button columns simultaneously. It first checks if the mouse event occurred over a button (\textit{my\%20 < 15}), assuming that each button is 15 pixels high, and buttons are spaced in 20 pixel intervals. It then checks if the event occurred in the left or right button column, and determines which of the buttons was pressed (\textit{bt}). Finally, the \textit{oapiProcessMFDButton} function is called with the appropriate parameters, to allow Orbiter to respond to the MFD request.\\
\\
The bottom row of buttons is processed similarly:

\begin{lstlisting}
	...
	case AID_BBUTTONS:
		if (mx < 20)
			oapiToggleMFD_on( MFD_LEFT );
		else if (mx >= 30 && mx < 50)
			oapiSendMFDKey( MFD_LEFT, OAPI_KEY_F1 );
		else if (mx > 60)
			oapiSendMFDKey( MFD_LEFT, OAPI_KEY_GRAVE );
		return true;
	...
\end{lstlisting}

\noindent
where \textit{oapiToggleMFD\_on} switches the MFD on/off, and the \textit{oapiSendMFDKey} commands trigger the default actions of displaying the key commands and the MFD mode list.\\
\\
Of course, the values of the various mouse x and y values in an actual implementation will depend on the geometry of the individual MFD layout. You could even define each single button as a separate area, but this will generally result in less efficient code.\\
\\
Finally, the MFD buttons need to respond to redraw events, to reflect the change of button labels (for example, when the MFD mode changes). Note that the MFD display area itself is automatically updated by Orbiter and therefore doesn't need to implement a redraw response.

\begin{lstlisting}
bool MyVessel::clbkPanelRedrawEvent( int id, int event, SURFHANDLE surf )
{
	switch (id)
	{
		case AID_LBUTTONS:
		case AID_RBUTTONS:
			side = (id == AID_LBUTTONS ? 0 : 1);
			hDC = oapiGetDC( surf );
			for (int bt = 0; bt < 6; bt++)
			{
				if (label = oapiMFDButtonLabel( MFD_LEFT, bt + side * 6) )
				TextOut( hDC, 5, 2 + 20 * bt, label, strlen( label ) );
				else break;
			}
			oapiReleaseDC( surf, hDC );
			return true;
		case ...
	}
	return false;
}
\end{lstlisting}

\noindent
This uses the \textit{oapiMFDButtonLabel} function to retrieve the label text for each of the buttons (button labels consist of 1 to 3 characters). The redraw function can be customised to reflect the style in which the button labels are displayed (for example by changing the font size or colour).\\
\\
Note that the bottom row of buttons does not necessarily need to implement a redraw method, because their labels never change.


\subsubsection{Multiple panels}
To implement multiple panels for a vessel, the \textit{clbkLoadPanel} method must load different panels depending on the provided panel \textit{id}, and each of the panels must define its connectivity to neighbouring panels via the \textit{oapiSetPanelNeighbours} function.\\
\\
Example: If your vessel supports a main panel, an overhead and a left side panel, the structure of the overloaded \textit{clbkLoadPanel} could look like this:

\begin{lstlisting}
bool MyVessel::clbkLoadPanel( int id )
{
	switch (id)
	{
		case 0:// main panel
			oapiRegisterPanelBackground( LoadBitmap( hDLL,
				MAKEINTRESOURCE(IDB_PANEL0) ) );
			oapiSetPanelNeighbours( 2, -1, 1, -1 );
			// register areas for panel 0 here
			break;
		case 1:// overhead panel
			oapiRegisterPanelBackground( LoadBitmap( hDLL,
				MAKEINTRESOURCE(IDB_PANEL1) ) );
			oapiSetPanelNeighbours( -1, -1, -1, 0 );
			// register areas for panel 1 here
			break;
		case 2:// left side panel
			oapiRegisterPanelBackground( LoadBitmap( hDLL,
				MAKEINTRESOURCE(IDB_PANEL2) ) );
			oapiSetPanelNeighbours( -1, 0, -1, -1 );
			// register areas for panel 2 here
			break;
	}
	return true;
}
\end{lstlisting}

\noindent
Each panel must register its own background bitmap via the \textit{oapiRegisterPanelBackground} function.\\
\\
In a vessel that defines multiple panels, the user can switch between them by using \Ctrl\DArrow\UArrow\RArrow\LArrow keys. Orbiter must know the relative location of bitmaps to each other, so that the correct panel can be loaded. This connectivity is provided by the \textit{oapiSetPanelNeighbours} function. This function tells Orbiter which panels are to the left, right, top and bottom of the current panel. A value of -1 indicates that no panel is located at that side.\\
\\
\textbf{Important}: All the panel id's defined during \textit{oapiSetPanelNeighbours} must be supported by \textit{clbkLoadPanel}. For example, if panel 0 calls \textit{oapiSetPanelNeighbours( 2, -1, 1, -1 )}, then panels 1 and 2 must be handled by \textit{clbkLoadPanel}.\\
\\
All panels must call the \textit{oapiSetPanelNeighbours} function, otherwise there is no way for the user to switch back to a different panel. Panel connectivities should usually be reciprocal, i.e. if panel 0 defines panel 1 as its top neighbour, then panel 1 should define panel 0 as its bottom neighbour. If only a single panel (panel 0) is supported, calling \textit{oapiSetPanelNeighbours} is not necessary.


\subsection{Designing virtual cockpits}
\label{ssec:vc_design}
A virtual cockpit requires a 3-D mesh representation. In principle it is possible to add the cockpit directly to the mesh used to represent the vessel in external views, and flag this mesh to be visible both in external and cockpit views (via the \textit{SetMeshVisibility} method), but in general it is more efficient to design a separate mesh for the cockpit which is visible only in virtual cockpit view mode (using \textit{SetMeshVisibility} with the \textit{MESHVIS\_VC} flag). Make sure that the cockpit mesh is consistent with the external mesh. A good way to achieve this is by building the VC together with the external mesh in your 3D design program, but exporting the cockpit and the external parts to separate mesh files.

\begin{figure}[H]
	\centering
	\subfigure{\includegraphics[width=0.99\textwidth]{dg_vc.png}}
	\caption{A view of the DG virtual cockpit.}
\end{figure}

\noindent
To make the VC mode available in your mesh class, you must overload the \textit{clbkLoadVC} method of the \textit{VESSEL2} class:

\begin{lstlisting}
bool MyVessel::clbkLoadVC( int id )
{
	...
}
\end{lstlisting}

\noindent
This will allow the user to switch to VC mode with the \keystroke{F8} key. The \textit{id} parameter is currently always 0. Eventually it will allow to select different cockpit positions.\\
\\
In the body of \textit{clbkLoadVC}, you can define camera parameters:

\begin{lstlisting}
bool MyVessel::clbkLoadVC( int id )
{
	SetCameraOffset( _V( 0, 1.5, 6.0 ) );
	SetCameraDefaultDirection( _V( 0, 0, 1 ) );
	SetCameraRotationRange( RAD * 120, RAD * 120, RAD * 70, RAD * 70 );
	SetCameraShiftRange( _V( 0, 0, 0.1 ), _V( -0.2, 0, 0 ), _V( 0.2, 0, 0 ) );
	...
}
\end{lstlisting}

\noindent
\textit{SetCameraOffset} defines the camera (or pilot eye) position in the vessel coordinate frame. \textit{SetCameraDefaultDirection} defines the default direction the pilot is looking toward, \textit{SetCameraRotationRange} defines how far he can turn his head left right, up and down, and \textit{SetCameraShiftRange} allows to simulate the pilot 'leaning' forward, left or right, for example to get a better view out of a window.\\
\\
Note that you only need to define these camera parameters here if they change between different cockpit view modes. If all camera modes use the same parameters, they can be defined globally in the overloaded \textit{clbkSetClassCaps} method.\\
\\
Once you have implemented the \textit{clbkLoadVC} method thus far and defined the VC mesh, you should be able to compile the module and test the virtual cockpit mode in Orbiter. Try rotating the view with \Alt\DArrow\UArrow\RArrow\LArrow keys, and 'leaning' with \Ctrl\Alt\DArrow\UArrow\RArrow\LArrow keys. When you are satisfied with the camera parameters, you can proceed to activate VC areas.


\subsubsection{Defining active VC areas}
As with 2-D panels, virtual cockpit areas must be activated to allow dynamic updates or to respond to user input. This is how an active area is defined in \textit{clbkLoadVC}:

\begin{lstlisting}
bool MyVessel::clbkLoadVC( int id )
{
	...
	SURFHANDLE tex = oapiGetTextureHandle( vcmesh, 10 );
	oapiVCRegisterArea( AID_BUTTON, _R( 0, 0, 20, 10 ), PANEL_REDRAW_ALWAYS,
		PANEL_MOUSE_LBDOWN, PANEL_MAP_BGONREQUEST, tex );
	oapiVCSetAreaClickmode_Spherical( AID_BUTTON, _V( 5, 3.3, 7.1 ), 2.5 );
	...
}
\end{lstlisting}

\noindent
As with 2-D panel area definitions, the first parameter of \textit{oapiVCRegisterArea} defines a unique identifier for the area (\textit{AID\_BUTTON} in this case). The next parameter defines a rectangular area (in pixel units) in a texture that is updated dynamically in a redraw event. If the area doesn't need to update any textures (e.g. because it only responds to mouse events, or because it provides visual feedback by modifying the mesh geometry), this parameter is ignored and can be set to \_R(0,0,0,0).\\
\\
The third parameter defines the events which trigger a redraw notification for the area. In this case we have set it to \textit{PANEL\_REDRAW\_ALWAYS}, i.e. we request a redraw notification at each simulation frame (typical for gauges whose displays change constantly). Note that unlike 2-D panels, the term 'redraw event' stands for any change in the visual representation of the area. This may consist of repainting a dynamic texture, but it could also mean a mesh group animation or direct editing of mesh vertices or texture coordinates.\\
\\
The fourth parameter defines the mouse events which trigger a notification for the area. They are used in the same way as 2-D panel areas, but an additional function call is required to define the mouse-sensitive area (see below).\\
\\
The fifth parameter defines the initial contents of the drawing bitmap passed to the redraw notification. It is used in the same way as for 2-D panels. However, if the redraw event does not update a dynamic texture, this \textit{must} be set to \textit{PANEL\_MAP\_NONE}.\\
\\
The last parameter is a handle to the dynamic texture passed to redraw notifications. In this example, we have obtained the texture handle from mesh group 10 of the VC mesh via a call to \textit{oapiGetTextureHandle}. Note that textures obtained for dynamic repainting must be labelled as dynamic in the mesh file (see next section). If the area does not need to redraw a texture during a redraw event, this parameter can be set to \textit{NULL}. In that case, there is a shorter version of \textit{oapiVCRegisterArea} for convenience which omits the second, fifth and sixth parameters.\\
\\
Unlike 2-D panels, the mouse-sensitive region of a VC area must be defined with a separate function call. In virtual cockpits, the sensitive region is a 3-D volume. Orbiter draws a virtual ray from the camera position through the screen point at which a mouse event occurred, and checks whether the ray intersects a mouse-sensitive volume. If so, the corresponding mouse event is generated.\\
\\
You can define either a spherical or a quadrilateral mouse-sensitive region. A spherical region is defined via the \textit{oapiVCSetAreaClickmode\_Spherical} method, where you specifiy the centre of the spherical region in the vessel frame of reference, and its radius. This will trigger a mouse event whenever the user clicks inside the projection of the sphere onto the simulation window.\\
\\
Quadrilateral (e.g. rectangular) regions are defined via the \textit{oapiVCSetAreaClickmode\_Quadrilateral} method, where you specify the four corners of the mouse-sensitive region in space. Again, this will trigger mouse events whenever the user clicks inside the projection of the sensitive area on the simulation window.\\
\\
Spherical regions are slightly more efficient for Orbiter to test, but quadrilateral regions return information about the relative position at which the mouse click occurred, so they are somewhat more versatile.


\subsubsection{Defining dynamic textures}
One way to provide information to the pilot in VC mode is by repainting the bitmaps used to texture VC mesh groups. For example, you can implement gauges and data displays in this way. You may even be able to re-use a panel area redraw method used for 2-D panels to update a VC texture, minimising the additional coding effort.\\
\\
\textbf{Important}: Orbiter can only draw into uncompressed textures. For this reason, textures which support dynamic repainting must be marked in the mesh file with a 'D' (\textit{dynamic}), e.g.

\begin{lstlisting}
...
Textures 2
tex1.dds
tex2_dyn.dds D
\end{lstlisting}

\noindent
Dynamic textures are less efficient than static ones, so you should try to keep them to a minimum. Collect all parts that require dynamic updates in one or few (small) texture files, and keep them apart from the static parts.


\subsubsection{The mouse event handler}
Whenever a mouse event occurs inside the mouse-sensitive volume of an active area, a notification is passed to your module. To respond to such events, you must overload the \textit{clbkVCMouseEvent} method of the \textit{VESSEL2} class.

\begin{lstlisting}
bool MyVessel::clbkVCMouseEvent( int id, int event, VECTOR3 &p )
{
	...
}
\end{lstlisting}

\noindent
where \textit{id} is the area identifier, and \textit{event} is the mouse event that triggered the notification (The VC notification uses the same event types as 2-D panels).\\
\\
Parameter \textit{p} returns some information about the mouse position at the event. The information returned depends on the area type for which the event was generated. For spherical regions, p.x contains the distance of the mouse position from the centre of the area, while p.y and p.z are not used. For quadrilateral regions, p.x and p.y contain the relative mouse x and y positions within the region, where the top left corner of the region has coordinates (0,0), and the bottom right corner has coordinates (1,1). This allows to define differentiated responses depending on where in the region the event occurred, similar to the procedure in 2-D panel regions.\\
\\
Inside \textit{clbkVCMouseEvent}, your code must check the area id and perform the appropriate actions:

\begin{lstlisting}
bool MyVessel::clbkVCMouseEvent( int id, int event, VECTOR3 &p )
{
	switch (id)
	{
		case AID_BUTTON:
			DoProcessButtonPress(...);
			return true;
		case ... // place response to other areas here
	}
	return false;
}
\end{lstlisting}


\subsubsection{The redraw event handler}
Any active areas which specified a redraw flag other than \textit{PANEL\_REDRAW\_NEVER} during initialisation, will trigger redraw notifications for the appropriate events. Your code needs to overload the \textit{clbkVCRedrawEvent} method of the \textit{VESSEL2} class to respond to those events.

\begin{lstlisting}
bool MyVessel::clbkVCRedrawEvent( int id, int event, SURFHANDLE surf )
{
	...
}
\end{lstlisting}

\noindent
where \textit{id} is the area identifier, \textit{event} is the redraw event that triggered the notification, and \textit{surf} is a handle to the dynamic texture to be redrawn. \textit{surf} may be \textit{NULL} if you didn't specify a texture during the area initialisation.\\
\\
Inside \textit{clbkVCRedrawEvent}, check the area id and perform the appropriate redraw action for that area. Typically, this will be one of the following:

\begin{itemize}
\item Repainting the dynamic texture passed to the notification handler. This is done in the same way as repainting 2-D panel areas. In fact, you may even be able to re-use the same code. Repainting textures is a good way to update displays and instrument gauges.
\item Animating a mesh group. This can be used to simulate flipping a switch or pushing a lever. See section \ref{ssec:def_anim_sec} for details on animations.
\item Editing a mesh. You can use the \textit{oapiMeshGroup} function to access the vertices of a mesh group, and edit the vertex positions or texture coordinates. Editing texture coordinates may be a good alternative to redrawing a texture if the texture is to switch between discrete pre-defined states.
\end{itemize}

\noindent
\textit{clbkVCRedrawEvent} should return \textit{true} only if you have modified the dynamic texture. If the texture was not modified, or is undefined, the function should return \textit{false}.


\subsubsection{Defining MFDs in the virtual cockpit}
To define a multifunctional display inside a virtual cockpit, you need to perform the following steps:\\
\\
Create a new group in the mesh consisting of a flat square area (defined by 4 vertices and 2 triangles). This is going to be the MFD display. The texture coordinates of the vertices should be: top left corner: (0,0), top right corner: (1,0), bottom left corner: (0,1) and bottom right corner: (1,1). Set 'TEXTURE 0' and 'FLAG 3' for this group. This will exclude the group from normal rendering (Orbiter uses a special render pass for MFDs). You can select a material of your choice. A material with specular reflection will produce a 'glass surface' effect.\\
\\
In \textit{clbkLoadVC}, define the MFD display with \textit{oapiVCRegisterMFD}:

\begin{lstlisting}
bool MyVessel::clbkLoadVC( int id )
{
	...
	static VCMFDSPEC mfds_left = {1, 100};
	oapiVCRegisterMFD( MFD_LEFT, &mfds_left );
	...
}
\end{lstlisting}

\noindent
\textit{VCMFDSPEC} is a structure which contains the mesh index and group index of the MFD display group defined in the previous step. \textit{oapiVCRegisterMFD} registers this group as an MFD display (in this case, the left MFD).\\
\\
Next, you need to define the MFD control buttons. How you implement them is mostly up to you. Typically, you define each button as a rectangle and collect all rectangles into a single mesh group. Reserve space on a dynamic texture for drawing the button labels, and set the texture coordinates for the button rectangles accordingly.\\
\\
Then you define an active area for each button to receive mouse events (but no redraw events). You also define a dummy area for redraw events. Pass the dynamic texture handle reserved for that purpose to the redraw area. This could look as follows:

\begin{lstlisting}
bool MyVessel::clbkLoadVC( int id )
{
	...
	oapiVCRegisterArea( AID_LBUTTONS, _R( 0, 0, 20, 100 ), PANEL_REDRAW_USER,
		PANEL_MOUSE_IGNORE, PANEL_MAP_BACKGROUND, tex );
	oapiVCRegisterArea( AID_RBUTTONS,_R( 20, 0, 40, 100 ), PANEL_REDRAW_USER,
		PANEL_MOUSE_IGNORE, PANEL_MAP_BACKGROUND, tex );
	for (i = 0; i < 6; i++)
	{
		oapiVCRegisterArea( AID_LBUTTON1 + i, PANEL_REDRAW_NEVER,
			PANEL_MOUSE_LBDOWN | PANEL_MOUSE_LBPRESSED );
		oapiVCSetAreaClickmode_Spherical( AID_LBUTTON1 + i,
			_V( 0.2, 0.1 - i * 0.02, 2.0 ), 0.01 );
		oapiVCRegisterArea( AID_RBUTTON1 + i, PANEL_REDRAW_NEVER,
			PANEL_MOUSE_LBDOWN | PANEL_MOUSE_LBPRESSED );
		oapiVCSetAreaClickmode_Spherical( AID_RBUTTON1 + i,
			_V( 0.4, 0.1 - i * 0.02, 2.0 ), 0.01 );
	}
	...
}
\end{lstlisting}

\noindent
You should also define mouse-active areas for the three bottom MFD buttons.\\
\\
In the mouse event handler, trap any mouse clicks on the MFD buttons and pass them to the \textit{oapiProcessMFDButton} function:

\begin{lstlisting}
bool MyVessel::clbkVCMouseEvent( int id, int event )
{
	if (id >= AID_LBUTTON1 && id < AID_LBUTTON1 + 12)
	{
		oapiProcessMFDButton( MFD_LEFT, id - AID_LBUTTON1, event );
		return true;
	}
	...
	return false;
}
\end{lstlisting}

\noindent
In the redraw event handler, trap MFD button redraw requests and redraw the buttons as required:

\begin{lstlisting}
bool MyVessel::clbkVCRedrawEvent( int id, int event, SURFHANDLE surf )
{
	switch (id)
	{
		case AID_LBUTTONS:
			RedrawMFDButtons( surf, MFD_LEFT, 0 );
			return true;
		case AID_RBUTTONS:
			RedrawMFDButtons( surf, MFD_RIGHT, 0 );
			return true;
		case ...
	}
}
\end{lstlisting}

\noindent
where \textit{RedrawMFDButtons} is assumed to be a locally defined function performing the redraw action. You may be able to re-use the same method used for drawing the MFD buttons in the 2-D panel (see section \ref{sssec:def_panel_mfd}).\\
\\
Finally, trigger a redraw event in the body of the MFD mode change callback notification.

\begin{lstlisting}
void MyVessel::MFDMode( int mfd, int mode )
{
	switch (mfd)
	{
		case MFD_LEFT:
			oapiTriggerVCRedrawArea( 0, AID_LBUTTONS );
			oapiTriggerVCRedrawArea( 0, AID_RBUTTONS );
			break;
		case ...
	}
}
\end{lstlisting}


\subsubsection{Defining the HUD in the virtual cockpit}
< to be completed >


\end{document}
